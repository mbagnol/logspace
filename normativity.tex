We are going to define the way observations accept and reject words.
Indeed there is an issue with the notion of representation: an observation is an element of $\,\ualg\,$ and can therefore only interact with \emph{representations} of a word, and there are many possible representation of the same word (in definition \ref{words}, different choices of closed terms lead to different representations).

Therefore one has to ensure that acceptation or rejection is independent of the representation, so that the construction makes sense.

\define[automorphism]
	{An \emph{automorphism} of a $\ast$-algebra $\,\C A\,$ is a linear application $\,\varphi\,:\: \C A\rightarrow\C A\,$ such that for all~$\,F,G\in\C A\,$
	
\vbox{	\begin{itemize}
		\item $\varphi(FG)=\varphi(F)\varphi(G)$
		\item $\varphi(F^\dagger)=\varphi(F)^\dagger$
		\item $\varphi$ is injective
	\end{itemize}}
	If there is a partial isometry $\,U\in \C A\,$ such that $\,\varphi(a)=U^\dagger aU\,$ for all $\,a\,$, then we say that $\,\varphi\,$ is an \emph{inner} automorphism.\nolinebreak
}

\define[normative pair]
	{\label{normpair}A pair $\,(\C A,\C B)\,$ of $\ast$-algebras is a \emph{normative pair} whenever any automorphism $\,\varphi\,$ of $\,\C A\,$ can be extended into an automorphism $\,\overline\varphi\,$ of $\:\alg^\dagger(\C A \cup\C B)\:$ such that $\,\overline\varphi(F)=F\,$ for any $\,F\in \C B\,\subseteq\,\alg^\dagger(\C A \cup\C B)\,$.
}

The two following theorems set the basis for a notion of acceptation/rejection independent of the representation of a word.

\smallskip
%\notation{Given a finite set of closed terms $\,S\,$, we write $\,\displaystyle1_S:=\sum_{\TT s\in S} s \flow s\,$.}

\theorem[automorphic representations]
	{\label{automorphic}Any two representations $\,W(t_i),W(u_i)\,$ of a word $\,W\,$ over $\,\Sigma\,$ are automorphic: there exists an automorphism $\,\varphi\,$ of $\,(\closed)^1\,$ such that \[\,(\Id_\ualg\tensor\varphi)\big(W(t_i)\big)=W(u_i)\,\]
}
\vspace{-5mm}
\proof{Take as $\,\varphi\,$ the inner automorphism associated with the isometric wiring

\smallskip
~\hfill$\displaystyle\sum_i t_i\p x \flow u_i\p x \:\in\, (\closed)^1$\hfill~
}

\smallskip
\theorem[nilpotency and normative pairs]
	{\label{normative}Let $\,(\C A,\C B)\,$ be a normative pair, $\,F\in \C A\,$, $\,G\in \C B\,$ and $\,\varphi\,$ an automorphism of $\,\C A\,$, then $\,GF\,$ is nilpotent if and only if $\,G\,\varphi(F)\,$ is nilpotent.
}

\proof{Let $\,\overline\varphi\,$ be the extension of $\,\varphi\,$ as in definition \ref{normpair}. We have for all $\,n\geq 1\,$ that $\,(G\varphi(F))^n=(\overline\varphi(G)\overline\varphi(F))^n=(\overline\varphi(GF))^n=\overline\varphi((GF)^n)\,$.

By injectivity of $\,\overline\varphi\,$, $\,(\varphi(G)F)^n=0\,$ if and only if $\,(GF)^n=0\,$.
}

\corollary[independance]
{\label{indep}If $\,\big((\C \closed)^1,\C B\big)\,$ is a normative pair, $\,W\,$ a word over $\,\Sigma\,$ and $\,F\in \closed\tensor\Sigma\tensor\IO\tensor\C B\,$, then $\,FW(t_i)\,$ is nilpotent for one choice of $\,t_i\,$ if and only if it is nilpotent for all choices of $\,t_i\,$.}

%Now we can use the tools from section \ref{actions} to build a particular normative pair.

The word and observation algebras we introduced earlier can be shown to form a normative pair.

\theorem[a normative pair]
{For any $\,\Sigma\,$, $\,\big((\closed)^1,\C S\big)\,$ is a normative pair.}

\proof[sketch]{By simple computations, the set
\[\,\C A:=\vect\set{\sigma F}{\sigma \in \C S \mbox{ and }\, F\in (\closed)^\infty}\,\]
 can be shown to be a $\ast$-algebra, so that $\,\alg^\dagger(\C S\cup(\closed)^1)=\alg^\dagger(\C S\cup(\closed)^\infty)=\C A\,$.

Now, if $\,\varphi\,$ is an automorphism of $\,(\closed)^1\,$, it can be written as $\,\varphi(F\tensor 1)=\psi(F)\tensor 1\,$ with $\,\psi\,$ an automorphism of $\,\closed\,$.
We define for $\,F=F_1\tensor\cdots\tensor F_n\tensor 1\in(\closed)^n\,$, $\,\tilde\phi(F):=\psi(F_1)\tensor\cdots\tensor \psi(F_n)\tensor 1\,$ which extends into an automorphism of $\,(\closed)^\infty\,$ by linearity.
Finally, we extend $\,\tilde\varphi\,$ to $\,\C A\,$ by $\,\overline\varphi(\sigma F):=\sigma\tilde\varphi(F)\,$
}

\medskip
\remark{Here we sketched a direct proof for brevity, but this can also be shown by involving a little more mathematical structure (actions of permutations on the unbounded tensor and crossed products) which would give a more synthetic proof.}

\medskip
Therefore the notion of the language recognized by an observation makes sense, \textit{via} corollary \ref{indep}.

\define[language of an observation]
	{Let $\,\phi\in\C O_\Sigma\,$ be an observation over $\,\Sigma\,$.
	The \emph{language recognized by $\,\phi\,$} is the following set of words over $\,\Sigma\,$
	\vspace{-1mm}
	\[\lang(\phi):=\set{W\:\mbox{word over}\,\Sigma}{\phi\,W(t_i)\,\mbox{nilpotent for any choice of}\,(t_i)}\]
	}
	\vspace{-8mm}
