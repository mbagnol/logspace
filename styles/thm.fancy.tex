%%%%%%%%%%%%%%%%%%%%%%%%%%%%%%%%%%%%%%%%%%%%%%%%%%%%%%%%%%%%%%%%%%%%%%%%%%%%%%
%% BARRES LATERALES

\newenvironment{leftbar2}
	{\def\FrameCommand{\hspace{12pt}\vrule width 2pt \hspace{7pt}}
		\MakeFramed {\advance\hsize-\width \FrameRestore}
	}
	{\endMakeFramed}

\newenvironment{leftbar1}
	{\def\FrameCommand{\hspace{10pt} \vrule width 0.6pt \hspace{7.7pt}}
		\MakeFramed {\advance\hsize-\width \FrameRestore}
	}
	{\endMakeFramed}

\newenvironment{leftbar11}
	{\def\FrameCommand{\hspace{10pt} \vrule width 0.6pt \hspace{-1.8pt} \vrule width 0.6pt \hspace{6.3pt}}
		\MakeFramed {\advance\hsize-\width \FrameRestore}
	}
	{\endMakeFramed}

\newenvironment{leftbar0}
	{\def\FrameCommand{\hspace{21pt}}
		\MakeFramed {\advance\hsize-\width \FrameRestore}
	}
	{\endMakeFramed}

%%%%%%%%%%%%%%%%%%%%%%%%%%%%%%%%%%%%%%%%%%%%%%%%%%%%%%%%%%%%%%%%%%%%%%%%%%%%%%
%% STYLES

%\newtheoremstyle{general}% name of the style to be used
%  {12pt}% measure of space to leave above the theorem. E.g.: 3pt
%  {12pt}% measure of space to leave below the theorem. E.g.: 3pt
%  {\normalfont}% name of font to use in the body of the theorem
%  {0pt}% measure of space to indent
%  {\bf}% name of head font
%  {}% punctuation between head and body
%  { }% space after theorem head; " " = normal interword space
%  {}% Manually specify head

%\theoremstyle{general}

\newtheorem{theorm}{Theorem}%[section]
\newtheorem{propositin}[theorm]{Proposition}
\newtheorem{definitin}[theorm]{Definition}
\newtheorem{lema}[theorm]{Lemma}
\newtheorem{corollay}[theorm]{Corollary}

%français
\newtheorem{theorèe}[theorm]{Théorème}
\newtheorem{définitin}[theorm]{Définition}
\newtheorem{leme}[theorm]{Lemme}
\newtheorem{corollaie}[theorm]{Corollaire}


%%%%%%%%%%%%%%%%%%%%%%%%%%%%%%%%%%%%%%%%%%%%%%%%%%%%%%%%%%%%%%%%%%%%%%%%%%%%%%
%% FONCTIONS


\newcommand{\optname}[1]{\ifthenelse{\equal{#1}{}}{\textbf.\\}{\!\!\textbf{(#1).\\}}}
\newcommand{\optparen}[1]{\ifthenelse{\equal{#1}{}}{}{~(#1)}}

\newcommandx{\define}[2][1=]{
%\vbox{
			\setlength{\parindent}{0pt}
	\begin{definitin}\optname{#1}
%		\vspace{-7pt}
%		\begin{leftbar1}
			\vspace{0.7pt}
			#2%\nolinebreak
			\vspace{0.7pt} 
%		\end{leftbar1} 
	\end{definitin}
%	}
}


\renewcommandx{\theorem}[2][1=]{
%\vbox{
			\setlength{\parindent}{0pt}
	\begin{theorm}\optname{#1}
%		\vspace{-7pt}
%		\begin{leftbar2}
			\vspace{0.7pt}
			#2%\nolinebreak
			\vspace{0.7pt} 
%		\end{leftbar2} 
	\end{theorm}
%	}
}


\renewcommandx{\lemma}[2][1=]{
%\vbox{
			\setlength{\parindent}{0pt}
	\begin{lema}\optname{#1}
%		\vspace{-7pt}
%		\begin{leftbar11}
			\vspace{0.7pt}
			\noindent
			#2%\nolinebreak
			\vspace{0.7pt} 
%		\end{leftbar11}
	\end{lema}
%	}
}



\renewcommandx{\corollary}[2][1=]{
%\vbox{
			\setlength{\parindent}{0pt}
	\begin{corollay}\optname{#1}
%		\vspace{-7pt}
%		\begin{leftbar1}
			\vspace{0.7pt}
			\noindent
			#2%\nolinebreak
			\vspace{0.7pt} 
%		\end{leftbar1} 
	\end{corollay}
%	}
}

\renewcommandx{\proof}[2][1=]{\noindent\textit{Proof\optparen{#1}}. #2\qed}

