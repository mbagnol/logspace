We now use the notions we just saw to build an algebra with a product based on unification.
Let us start with a monoid with a partially defined product, that is the basis of the construction.

\define[flows]
	{\label{def_flow}A flow is an oriented pair written $\,t\flow u\,$ of terms such that $\,\var(t)=\var(u)\,$.
	
%	\smallskip
	Flows are considered up to renaming: for any renaming $\,\alpha\,$ we have $\,t\flow u\,=\,t.\alpha\flow u.\alpha\,$.
	
%	\smallskip
	We will write $\,\flows\,$ the set of (equivalence classes of) flows.
	
	We set $\,1:=x\flow x\,$ and $\,(t\flow u)^\dagger:=u\flow t\,$ so that $\,(.)^\dagger\,$ is an involution $\,\flows\,$.
	}

A flow $\,t\flow u\,$ can be thought of as a \texttt{ `match ... with u -> t' } in a ML-style language.
The composition of flows follows intuition.

\define[product of flows]
	{Let $\,u\flow v\in \flows\,$ and $\,t\flow w\in \flows\,$.
	Suppose we have chosen two representatives of the renaming classes such that their sets of variables are disjoint.

%\smallskip
The product of $\,u\flow v\,$ and $\,t\flow w\,$ is defined if $\,v,t\,$ are unifiable with MGU $\,\theta\,$ and in that case
\[(u\flow v)(t\flow w)\,:=\:u.\theta \flow w.\theta\]
\vspace{-5mm}
}

\define[action on closed terms]
	{\label{flow-action}If $\,t\in \closed\,$ is a closed term, $\,(u\flow v)(t)\,$ is defined whenever $\,t\,$ and $\,v\,$ are unifiable, with MGU $\,\theta\,$, in that case $\,(u\flow v)(t):=u.\theta\,$
}

\remark{The condition on variables ensures that the result is a closed term (because \hbox{$\,\var(u)\subseteq\var(v)\,$}) and that the action is injective on its domain of definition (because \hbox{$\,\var(v)\subseteq\var(u)\,$}).

Moreover, the action is compatible with the product of flows: $\,l(k(t))=(l\,k)(t)\,$ and both sides are defined at the same time.
}

\medskip
By adding a formal element $\,\bot\,$ (representing the failure of unification) to the set of flows, one could turn their product in a completely defined operation, making $\,\C F\,$ an \emph{inverse monoid}. However, we will need to consider the algebra of \emph{sums} of flows that is easily defined directly from the partially defined product.


\define[wirings]
	{We call \emph{wirings} $\BB C$-linear combinations (formally: the set of almost-everywhere null functions from $\,\flows\,$ to $\,\BB C\,$) of elements of $\,\flows\,$, endowed with:
	\[\bigg(\sum_i \lambda_i\,l_i\bigg) \bigg(\sum_j \mu_j\,k_j\bigg):=
		\:\sum_{\mathclap{\substack{i,j \,\text{ such that} \\ (l_ik_j)\,\text{is defined}}}}\!\!\lambda_i\mu_j(l_i\,k_j)\]
		and
	\[\bigg(\sum_i \lambda_i\,l_i\bigg)^\dagger:=\:\sum_i \,\overline\lambda_i\,l_i^\dagger\]
	
	We write $\,\ualg\,$ the set of wirings and refer to it as the \emph{unification algebra}.
}

\remark{$\ualg\,$ is indeed a unital $\ast$-algebra: the only delicate point would be the associativity of the product, which is a consequence of lemma \ref{part-unif}.}

\define[partial isometries and orthogonal projections]
	{\label{piso}A \emph{partial isometry} is a wiring $\,U\,$ satisfying $\,UU^\dagger U=U\,$.% We write $\,\partialisos\,$ the set of \emph{concrete} partial isometries. 

An \emph{orthogonal projection} is a wiring $\,U\,$ satisfying $\,U^2=U\,$ and $\,U^\dagger=U\,$.
}


\smallskip
While $\,\ualg\,$ offers the algebraic background to work in, we will need to consider particular kind of wirings to study computation.

\define[concrete and isometric wirings]
{A wiring is \emph{concrete} whenever it is a sum of flows with all coefficients equal to~$\,1\,$.

An \emph{isometric wiring} is a concrete wiring that is also a partial isometry.

\smallskip
Given a set of wirings $\,E\,$ we will write $\,E^+\,$ the set of concrete wirings of $\,E\,$.
}

Isometric wirings enjoy a direct characterisation.

\theorem[isometric wirings]
{The isometric wirings are exactly the wirings of the form
	$\,\sum_i \, u_i \flow t_i\,$
	with the $\,u_i\,$ pairwise disjoint (definition \ref{disjoint}) and the $\,t_i\,$ pairwise disjoint.
}

It will be useful to consider the action of wirings on closed terms. For this purpose we extend definition \ref{flow-action} to wirings.

\define[action on closed terms]
	{Let $\,\closedh\,$ be the free $\BB C$-vector space over $\,\closed\,$.
	
%	\smallskip
	Wirings act on base vectors of $\,\closedh\,$ the following way
	\[\bigg(\sum_i \lambda_i\,l_i\bigg)(t) :=\!\sum_{\mathclap{\substack{i \,\text{ such that} \\ l_i(t)\,\text{ is defined}}}}\!\!\lambda_i \big(l_i(t)\big) \ \ \in\: \closedh\]
	which extends by linearity into an action on the whole $\,\closedh\,$.
}

Note that isometric wirings have a particular behaviour in terms of this action.

\lemma[isometric action]
{\label{lem_isom}Let $\,F\,$ be a isometric wiring and $\,t\,$ a closed term. We have that $\,F(t)\,$ and $\,F^\dagger(t)\,$ are either $\,0\,$ or another closed term $\,t'\,$ (seen as an element of $\,\closedh\,$).}
