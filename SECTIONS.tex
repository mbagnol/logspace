%%%%%%%%%%%%%%%%%%%%%%%%%%%%%%%%%%%%%%%%%%%%%%%%%%%%%%%%%% TITRE
%%%%%%%%%%%%%%%%%%%%%%%%%%%%%%%%%%%%%%%%%%%%%%%%%%%%%%%%%%


\title{Unification and Logarithmic Space}
\author{Clément Aubert \and Marc Bagnol\thanks{This work was partly supported by the ANR-10-BLAN-0213 Logoi and the ANR-11-BS02-0010 Récré.}}
\institute{Aix-Marseille Université, CNRS, I2M, UMR 7373, 13453 Marseille, France}
%\date{}

\maketitle


%%%%%%%%%%%%%%%%%%%%%%%%%%%%%%%%%%%%%%%%%%%%%%%%% TABLE


%\tableofcontents\thispagestyle{empty}


%%%%%%%%%%%% CHAPTERS and SECTIONS

\begin{abstract}
	We present an algebraic characterization of the complexity classes \textsc{Logspace} and \textsc{NLogspace} of deterministic and non-deterministic logarithmic space, using an algebra with a composition law based on unification. We show that computation can be modeled in it by means of specific subalgebras related finite permutations groups.

We also show that the construction can be naturally related with a notion pointer machines, which is another model of logarithmic space computation.
\end{abstract}


\section*{Introduction}
\addcontentsline{toc}{section}{Introduction}
\textbf{Proof theory and complexity theory.} Since the introduction of so-called light logics \cite{girard_light_1994}, several authors contributed to the topic of implicit complexity theory using tools coming from proof theory and more specifically linear logic \cite{girard_linear_1987}.

The study of geometrical properties of proofnets \cite{girard_linear_1987}, in terms for instance of stratitification of exponential boxes \cite{baillot_linear_2010} led to a clearer understanding of the time complexity of the cut-elimination procedure. Space complexity has also been studied from this perspective \cite{schopp_stratified_2007,gaboardi_logical_2008}.

\smallskip\noindent
\textbf{Geometry of Interaction.} As the study of cut-elimination has grown as a central topic in proof theory, its mathematical modelling became of great interest. The Geometry of Interaction \cite{girard_towards_1989} research program led to models of cut-elimination in terms of paths in proofnets \cite{asperti_paths_1994}, token machines \cite{laurent_token_2001} and operator algebras \cite{girard_geometry_1989}.
The relationship of these ideas with complexity theory has also been investigated \cite{schopp_space-efficient_2006,baillot_elementary_2001}.

The tools shaped by this approach have also been used to study directly complexity theory from an algebraic point of view \cite{girard_normativity_2012,aubert_characterizing_2012,seiller_logarithmic_2013}. These works are the main source of inspiration for this article.

\smallskip\noindent
\textbf{Unification.} The unification technique has been used \cite{girard_geometry_1995,baillot_elementary_2001,girard_three_lightings} to provide a setting where one can model the cut-elimination procedure in a finitary way: first-order terms with variables offer a way to manipulate infinite sets (of instances of terms) with a finite language.

It turns out that this language is expressive enough to encode various algebraic structures, including a notion of unbounded tensor product and a representation of finite permutation groups. These were provided by the use of the theory of von Neumann algebras \cite{girard_normativity_2012,aubert_characterizing_2012,seiller_logarithmic_2013}.

\smallskip\noindent
\textbf{Pointer machines.} The study of space efficient computations lead to consider pointer machines as a mathematical model of computing \cite{hofmann_pointer_2009}.%,hofmann_pure_2010}.

In \cite{girard_normativity_2012}, how operators simulate computation is briefly explained in terms of pointer machines, an idea that has been made precise by the later \cite{aubert_characterizing_2012,seiller_logarithmic_2013}.

%We introduce here a notion of pointer machine that is specifically designed to be easily translated into 

\smallskip\noindent
\textbf{Organisation of this article.} In Sect.\ref{sec_unification} we review some classical results on unification of first-order terms and use them to build the algebra that will constitute our computationnal setting.

We explain in Sect.\ref{sec_words} how words and computing devices (observations) can be modeled by particular elements of this algebra.
The way they interact to yield a notion of language recognized by an observation is described in Sect.\ref{sec_normativity}.

Finally, we show in Sect.\ref{sec_logspace} that our construction captures exactly logarithmic space computation, both deterministic and non-deterministic.

\section{The Unification Algebra}\label{sec_unification}
	\subsection{Unification}
	Unification can be generally thought of the study of formal solving of equations between terms.

This topic was introduced by Herbrand in his thesis \cite{herbrand_recherches_1930}, but became really widespread after the work of J.~A.~Robinson on automated theorem proving \cite{robinson_machine-oriented_1965}.
The unification technique is also at the core of the logic programming language \textsc{Prolog} and type inference for functional programming languages such as \textsc{CaML} and \textsc{Haskell}.

%\subsubsection*{Terms and substitutions}

Specifically, we will be interested in the following problem: 
\begin{center}
	\it Given two (first-order) terms,\\ can they be “made equal” by replacing their variables?
\end{center}

To make things more concrete, we set a specific set of terms for the rest of this article.

\define[terms]
	{We consider the following set of first-order terms
	
	\[\terms::=\ \ x,y,z,\:\dots\ |\ \TT a,\tt b,\tt c,\:\dots\ |\ \terms\p\terms\]
	
	where  $\,x,y,z,\:\dots\in \vars\,$ are variables and $\,\TT a,\TT b,\TT c,\:\dots\,$ are constants, while $\,\p\,$ is a binary function symbol.

\smallskip
For any $\,t\in\terms\,$, we will write $\,\var(t)\,$ the set of variables occuring in $\,t\,$.
We say that a term is closed whenever $\,\var(t)=\varnothing\,$, and denote the set of closed terms as $\,\closed\,$. 
}

\notation{We will write the binary function symbol as \emph{right associating} to avoid an excess of parenthesis: $\:t\p u\p v \::=\:t\p(u\p v) \,$}
%We also leave the $\,\p\,$ implicit $\,\TT c\p u\,$, so that: $\,\TT{lrc}\p\TT{llc}:=(\TT l \p\TT r \p\TT c)\p \TT l \p\TT l \p\TT c=(\TT l \p(\TT r \p\TT c))\p (\TT l \p(\TT l \p\TT c))\,$}

%Variables in terms are meant to be replaced when needed by a term of the language. This is implemented by the notion of substitution.

\define[substitution]
	{A substitution is a map $\,\theta:\:\vars \rightarrow \terms\,$ such that the set $\,\dom(\theta):=\{\:v\in \vars\:|\:\theta(v)\not=v\:\}\,$ (the \emph{domain} of $\,\theta\,$) is finite.
	A substitution with domain $\,\{\,x_1,\dots,x_n\,\}\,$ such that $\,\theta(x_1)=u_1\,,\,\dots\,,\,\theta(x_n)=u_n\,$ will be written as $\,\{\:x_1\mapsto u_1\,;\,\dots\,;\, x_n\mapsto u_n\:\}\,$.

\smallskip
	If $\,t\in \terms\,$ is a term we write $\,t.\theta\,$ the term $\,t\,$ where any occurence of any variable $\,x\,$ has been replaced by $\,\theta(x)\,$.
	
	\smallskip
	If $\,\theta=\{\:x_i\mapsto u_i\:\}$ and $\,\psi=\{\:y_j\mapsto v_j\:\}\,$, their \emph{composition} is defined as
\[\theta;\psi\::=\: 
	\{\:x_i\mapsto u_i.\psi\:\} \:\cup\:
	\{\:y_i\mapsto v_i \:|\;y_i \not\in \dom(\theta)\:\}\]
\vspace{-7mm}
}

\remark{The composition of substitutions is such that $\,t.(\theta;\psi)=(t.\theta).\psi\,$ holds.}

\define[vocabulary]{A \emph{renaming} is a substitution $\,\alpha\,$ such that $\,\alpha(\vars)\subseteq \vars\,$ and that is bijective.

\smallskip Two substitutions $\,\theta,\psi\,$ are equal \emph{up to renaming} if there exist a renaming $\,\alpha\,$ such that $\,\psi=\theta;\alpha\,$.
Two terms $\,t,u\,$ are equal \emph{up to renaming} if there exist a renaming $\,\alpha\,$ such that $\,t=u.\alpha\,$.

\smallskip A substitution $\,\psi\,$ is an \emph{instance} of $\,\theta\,$ if there exists a substitution $\,\sigma\,$ such that $\,\psi=\theta;\sigma\,$.

%\smallskip A substitution $\,\theta\,$ is \emph{idempotent} if $\,\theta;\theta=\theta\,$.
}

\theorem%[folklore]
	{\label{folklore}The following properties hold for any substitutions $\,\theta,\psi\,$:
	\begin{itemize}%[parsep=0pt]
		\item The only invertible substitutions are renamings.
%		\item Every substitution is equal up to renaming to an idempotent substitution.
		\item If $\,\theta\,$ is an instance of $\,\psi\,$ and $\,\psi\,$ is an instance of $\,\theta\,$, then they are equal up to renaming.
	\end{itemize}
}

%\subsubsection*{Unification problems}

To allow easier manipulation, the problem of unifying \emph{two} terms needs to be generalized into the problem of simultaneously unifying several pairs of terms.

\define[unification]
	{A \emph{unification problem} is finite set of pair of terms $\,P=\{\:t_i\uequ u_i\:\}\,$.
	
	It is said to be unifiable if it has a \emph{unifier}: a substitution $\,\theta\,$ such that
	\[u_i.\theta=t_i.\theta\ \text{ for all }\,i\]
		In particular, we say that two terms $\,t,u\,$ are \emph{unifiable} if there exists a substitution $\,\theta\,$ such that $\,t.\theta=u.\theta\,$.
		
	$\theta\,$ is a \emph{most general unifier (MGU) of $\,P\,$} if any other unifier is an instance of~$\,\theta\,$.
}
\vspace{-5mm}

\remark{It follows from Theorem \ref{folklore} that any two MGU of the same unification problem are equal up to renaming.}

\smallskip
We will be interested mostly in the weaker variant of unification where one can first rename terms so that they variables are distinct, we introduce therefore a specific vocabulary for it.

\define[disjointness and matching]
	{\label{disjoint}Two terms $\,t,u\,$ are \emph{matchable} if $\,t',u'\,$ are unifiable, where $\,t',u'\,$ are renamings of $\,t,u\,$ such that $\,\var(t')\cap\var(u')=\varnothing\,$.
	
	If two terms are not matchable, they are said to be \emph{disjoint}.
}

\example{$x\,$ and $\,\TT f\p x\,$ are not unifiable.
However they are matchable, as $\,x.\{\,x\mapsto y\,;\,y\mapsto x\,\}=y\,$ which is unifiable with $\,\TT f\p x\,$.

More generally, disjointness is stronger than non-unifyability.}

\medskip
The crucial feature of first-order unification is the (decidable) existence of most general unifiers for unification problems that have a solution.

\theorem[MGU]
{If a unification problem has a unifier, then it has a MGU.

Wether two terms are unifiable and, in case they are, finding a MGU is decidable.}


Let us note the following fact, that will be useful in the next section: solving a unification problem can be done incrementally.

\smallskip
\notation{If $\,P=\{\:t_i\uequ u_i\:\}\,$ is a unification problem and $\,\theta\,$ a substitution, we write $\,P.\theta:=\{\:t_i.\theta\,\uequ\, u_i.\theta\:\}\,$.}

\lemma[partial unification]
	{\label{part-unif}Let $\,P=Q\uplus R\,$ ($\,\uplus\,$ denotes disjoint union) be a unification problem.
	The following statements are equivalent:
	\begin{itemize}
		\item $\,P\,$ is unifiable
		\item $\,Q\,$ is unifiable with MGU $\,\theta\,$ and $\,R.\theta\,$ is unifiable with MGU $\,\psi\,$
	\end{itemize}
In that case, we have moreover that $\,\theta;\psi\,$ is a MGU of $\,P\,$.
}

%\proof{The algorithm from the above proof can be modified to treat first pairs of $\,Q\,$ at line 3 without affecting its behaviour. 
%}

%\subsubsection*{Complexity of unification}

%Herbrand's original procedure for solving unification was non-deterministic.

J.~A.~Robinson was the first to prove the existence of MGU for two unifiable terms and derived from this a deterministic unification procedure.
His original procedure was however quite inefficient, with potential exponential blowups in some cases. 

\smallskip
It turns out that the general unification problem can be solved in linear time, which was discovered independently in \cite{paterson_linear_1978} and \cite{martelli_unification_1976}, improving on the almost-linear algorithms thus far designed.

\smallskip
Unification was first thought to be a \textsc{NLogspace}-complete problem. However, an error was eventually found in the note \cite{lewis_unifiability_1982} claiming this fact, which eventually led to a determination of the intrinsic complexity of the problem: unification is indeed \textsc{Ptime}-complete under logarithmic space reductions \cite{dwork_sequential_1984}.

\medskip
In this article, we are concerned with a very much simpler case of the problem: the matching (definition \ref{disjoint}) of linear terms (\textit{ie.} where variables occur at most once).
This case can be solved in a space-efficient way.

\theorem[logarithmic space {\cite[lemma 20]{dwork_parallel_1988}}]
{\label{unif-logspace}Wether two linear terms $\,t,u\,$ with disjoint sets of variables are unifiable, and if so finding a MGU, can be computed in logarithmic space on a deterministic Turing machine.

\smallskip
Moreover, their MGU is of size at most the sum of the sizes of $\,t\,$ and $\,u\,$.
}

The lemma in the article actually states that the problem is in \textsc{NC}\up 1, a complexity class of parallel computations known to be included in \textsc{Logspace}.
	\subsection{Flows and Wirings}
	We now use the notions we just saw to build an algebra with a product based on unification.
Let us start with a monoid with a partially defined product, which will be the basis of the construction.

\define[flows]
	{\label{def_flow}A \emph{flow} is an oriented pair written $\,t\flow u\,$ with $\,t,u\in\terms\,$ such that $\,\var(t)=\var(u)\,$.
	
%	\smallskip
	Flows are considered up to renaming: for any renaming $\,\alpha\,$ we have $$\,t\flow u\,=\,t.\alpha\flow u.\alpha\,$$
	
%	\smallskip
	We will write $\,\flows\,$ the set of (equivalence classes of) flows.
	
	We set $\,\unit:=x\flow x\,$ and $\,(t\flow u)^\dagger:=u\flow t\,$ so that $\,(.)^\dagger\,$ is an involution of $\,\flows\,$.
	}

A flow $\,t\flow u\,$ can be thought of as a \texttt{ `match ... with u -> t' } in a ML-style language.
The composition of flows follows this intuition.

\define[product of flows]
	{Let $\,u\flow v\in \flows\,$ and $\,t\flow w\in \flows\,$.
	Suppose we have chosen two representatives of the renaming classes such that their sets of variables are disjoint.

%\smallskip
The \emph{product} of $\,u\flow v\,$ and $\,t\flow w\,$ is defined
%\footnote{This definition respects equivalence classes of flows because of the fact that MGU are equal up to renaming}
if $\,v,t\,$ are unifiable with MGU $\,\theta\,$ (the choice of a MGU does not matter because of the remark following Definition~\ref{def_unification}) and in that case
\[(u\flow v)(t\flow w)\,:=\:u.\theta \flow w.\theta\]
\vspace{-5mm}
}



\define[action on closed terms]
	{\label{flow-action}If $\,t\in \closed\,$ is a closed term, $\,(u\flow v)(t)\,$ is defined whenever $\,t\,$ and $\,v\,$ are unifiable, with MGU $\,\theta\,$, in that case $\,(u\flow v)(t):=u.\theta\,$
}

\remark{The condition on variables ensures that the result is a closed term (because \hbox{$\,\var(u)\subseteq\var(v)\,$}) and that the action is injective on its domain of definition (because \hbox{$\,\var(v)\subseteq\var(u)\,$}).

Moreover, the action is compatible with the product of flows: $\,l(k(t))=(l\,k)(t)\,$ and both sides are defined at the same time.
}

\medskip
By adding a formal element $\,\bot\,$ (representing the failure of unification) to the set of flows, one could turn the product in a completely defined operation, making $\,\C F\,$ an \emph{inverse monoid}. However, we will need to consider the algebra of \emph{sums} of flows that is easily defined directly from the partially defined product.


\define[wirings]
	{We call \emph{wirings} $\BB C$-linear combinations (formally: the set of almost-everywhere null functions from $\,\flows\,$ to $\,\BB C\,$) of elements of $\,\flows\,$, endowed with:
	\[\bigg(\sum_i \lambda_i\,l_i\bigg) \bigg(\sum_j \mu_j\,k_j\bigg):=
		\:\sum_{\mathclap{\substack{i,j \,\text{ such that} \\ (l_ik_j)\,\text{is defined}}}}\lambda_i\mu_j(l_i\,k_j)\]
	\vspace{-2mm}
	\[\text{and\qquad} \bigg(\sum_i \lambda_i\,l_i\bigg)^\dagger:=\:\sum_i \,\overline\lambda_i\,l_i^\dagger
	\qquad\scriptstyle\text{(where }\,\overline\lambda\,\text{ is the complex conjugate of }\,\lambda\,\text{)}
	\]
	
	We write $\,\ualg\,$ the set of wirings and refer to it as the \emph{unification algebra}.
}

\remark{Indeed, $\ualg\,$ is a unital $\ast$-algebra: the only delicate point would be the associativity of the product, which is a consequence of Lemma \ref{part-unif}.}

\define[partial isometries and orthogonal projections]
	{\label{piso}A \emph{partial isometry} is a wiring $\,U\in \ualg\,$ satisfying $\,UU^\dagger U=U\,$.% We write $\,\partialisos\,$ the set of \emph{concrete} partial isometries. 

An \emph{orthogonal projection} is a wiring $\,U\in \ualg\,$ satisfying $\,U^2=U\,$ and $\,U^\dagger=U\,$.
}


\smallskip
While $\,\ualg\,$ offers the algebraic background to work in, we will need to consider particular kind of wirings to study computation.

\define[concrete and isometric wirings]
{\label{concrete}A wiring is \emph{concrete} whenever it is a sum of flows with all coefficients equal to~$\,1\,$.

An \emph{isometric wiring} is a concrete wiring that is also a partial isometry.

\smallskip
Given a set of wirings $\,E\,$ we will write $\,E^+\,$ the set of concrete wirings of $\,E\,$.
}

Isometric wirings enjoy a direct characterization.

\proposition[isometric wirings]
{The isometric wirings are exactly the wirings of the form
	$\,\sum_i \, u_i \flow t_i\,$
	with the $\,u_i\,$ pairwise disjoint (Definition \ref{disjoint}) and the $\,t_i\,$ pairwise disjoint.
}

It will be useful to consider the action of wirings on closed terms.
For this purpose we extend Definition \ref{flow-action} to wirings.

\define[action on closed terms]
	{Let $\,\closedh\,$ be the free $\BB C$-vector space over $\,\closed\,$.
	
%	\smallskip
	Wirings act on base vectors of $\,\closedh\,$ in the following way
	\[\bigg(\sum_i \lambda_i\,l_i\bigg)(t) :=\!\sum_{\mathclap{\substack{i \,\text{ such that} \\ l_i(t)\,\text{ is defined}}}}\lambda_i \big(l_i(t)\big) \ \ \in\: \closedh\]
	which extends by linearity into an action on the whole $\,\closedh\,$.
}

Note that isometric wirings have a particular behavior in terms of this action.

\lemma[isometric action]
{\label{lem_isom}Let $\,F\,$ be an isometric wiring and $\,t\,$ a closed term.
We have that $\,F(t)\,$ and $\,F^\dagger(t)\,$ are either $\,0\,$ or another closed term $\,t'\,$ (seen as an element of $\,\closedh\,$).}

	\subsection{Permutations and Tensor Product}\label{permutation}
	We define now the representation in $\,\ualg\,$ of structures that provide enough expressivity to model computation.

Unbounded tensor products will allow to represent data of arbitrary size, and finite-support permutations will be used to manipulate these data.

\medskip
\notations{Given any set of wirings $\,E\,$, we write $\,\vect(E)\,$ the vector space
generated by $\,E\,$, \emph{ie.} the set of \emph{finite} linear combinations of elements of $\,E\,$ (for instance $\,\vect(\closed)=\closedh\,$).

<<<<<<< HEAD
Moreover, we set 
$\,\unitalg:=\set{\lambda\unit}{\lambda\in \BB C\,}\,$ (with $\,I=x\flow x\,$ as in Definition \ref{def_flow})
and
$\,u\sflow v:=\, u\flow v +v\flow u\,$.
=======
Moreover, we set, taking $\,\unit:=x\flow x\,$ as in \autoref{def_flow},
\[\unitalg:=\,\BB C.\unit=\vect\{\:\unit\:\}=\alg^\dagger\{\:\unit\:\}\qquad\quad
u\sflow v:=\, u\flow v +v\flow u\qquad\quad
>>>>>>> FETCH_HEAD
%\pflow u:=\,u\flow u

\smallskip
For brevity we write \enquote{$\ast$-algebra} instead of the more correct \enquote{$\ast$-subalgebra of $\,\ualg\,$}.
}

%Two important features of $\,\ualg\,$ for this work are the possibility of representing (unbounded) tensor products and finite-support permutations of $\,\BB N\,$.

\define[tensor product]
	{Let $\,u\flow v\,$ and $\,t\flow w\,$ be two flows.
	Suppose we have chosen representatives of these renaming classes that have their sets of variables disjoint. We define their \emph{tensor product} as
	$\,(u\flow v) \ptensor (t\flow w):=\: u\p t \flow v\p w\,$.
	The operation is extended to wirings by bilinearity.
	
	\smallskip
	Given two $\ast$-algebras $\,\C A,\C B\,$, we define their tensor product as the $\ast$-algebra
	\[\C A\ptensor \C B:= \,\vect\set{F\ptensor G}{F\in\C A,\:G\in \C B}\]
	\vspace{-7mm}
}

This actually defines an embedding of the algebraic tensor product of $\ast$-algebras into $\,\ualg\,$, which means in particular that $\,(F\ptensor G)(P\ptensor Q)=(FP)\ptensor(GQ)\,$. It ensures in particular that the $\,\ptensor\,$ operation indeed yields $\ast$-algebras.

\smallskip
\notation{Again, the $\,\ptensor\,$ is not associative. We carry on our convention and write it as \emph{right associating}: $\,\C A\ptensor\C B\ptensor\C C \::=\:\C A\ptensor(\C B\ptensor\C C) \,$.}
\vspace{-1mm}

\define[unbounded tensor]
	{\label{unbounded}%
	Let $\,\C A\,$ be a $\ast$-algebra. We define the $\ast$-algebras $\,\C A\tpower n\,$ for all $\,n\in \BB N\,$ as
	\vspace{-2mm}
	\[\C A\tpower 0:= \unitalg \quad\text{and}\quad \C A\tpower {n+1}:=\,\C A \ptensor \C A\tpower n\]
	\vspace{-3mm}
	and the $\ast$-algebra $\ \displaystyle\C A\tpower \infty:=\:\bigcup_{\mathclap{n\in \BB N}} \:\C A\tpower n\ $.% and $\ \displaystyle\C A^{\leq N}:=\:\bigcup_{n\in\{\,1,\dots,N\,\}}^N \:\C A^n\ $.
}

We will consider finite permutations, but allow them to be composed even when their domain of definition is not the same.

\smallskip
\notations{Let $\,\F S_n\,$ be the set of finite permutations over $\{1, \hdots, n\}$, if $\,\sigma\in \F S_n\,$, we define $\,\sigma_{+k}\in \F S_{n+k}\,$ as the permutation $\,\sigma\,$ extended to $\,\{\:1,\dots,n,\dots,n+k\:\}\,$ letting $\,\sigma_{+k}(n+i):=n+i\,$ for $\,i\in\{\,1,\dots,k\,\}\,$.

We also write $\,\unit_k:=\Id_{\{1,\dots,k\}}\in \F S_k\,$.}

\define[representation]
	{To a permutation $\,\sigma \in \F S_n\,$ we associate the flow
	\vspace{-1mm}
	\[[\sigma]:= \,x_1\p x_2\p\,\cdots\,\p x_n \p y\flow x_{\sigma(1)}\p x_{\sigma(2)}\p\,\cdots\,\p x_{\sigma(n)}\p y\]
	\vspace{-6mm}
%	We will write $\,\,$
	
%	We will omit the brackets when there is no risk of ambiguity.
}

<<<<<<< HEAD
A permutation $\,\sigma\in\F S_n\,$ will act on the first $\,n\,$ component of the unbounded tensor product (Definition \ref{unbounded}) by swapping them and leaving the rest unchanged.

The wirings $\,[\sigma]\,$ internalize this action: in the above definition, the variable $\,y\,$ at the end stands for the components that are not affected.

\smallskip
\example{Let $\,\tau\in\F S_2\,$ be the  permutation swapping the two elements of $\,\{1,2\}\,$ and $\,U_1\ptensor U_2\ptensor U_3\ptensor \unit \in \ualg\tpower 3\subseteq \ualg\tpower\infty\,$. 
=======
A permutation $\,\sigma\in\F S_n\,$ will act on the first $\,n\,$ components of the unbounded tensor product (Definition \ref{unbounded}) by swapping them and leaving the rest unchanged.

The wirings $\,[\sigma]\,$ internalize this action: in the above definition, the variable $\,y\,$ at the end stands for the components that are not affected.
>>>>>>> FETCH_HEAD

\smallskip
Then $\,[\tau](U_1\ptensor U_2\ptensor U_3\ptensor \unit)[\tau]^\dagger=U_2\ptensor U_1\ptensor U_3\ptensor \unit\,$.}

\proposition[representation]
	{\label{open-represent}For $\,\sigma\in \F S_n\,$ and $\,\tau\in\F S_{n+k}\,$ we have
	\vspace{-1mm}
	\[[\sigma_{+k}]=[\sigma][\unit_{n+k}]=[\unit_{n+k}][\sigma] \qquad
	[\sigma_{+k}\circ\tau]=[\sigma][\tau]\qquad
	 \text{and} \qquad
	 [\sigma^{-1}]=[\sigma]^\dagger
	\]
	\vspace{-5mm}
}

\define[permutation algebra]
	{For $\,n\in \BB N\,$ we set $\:[\F S_n]:=\,\set{[\sigma]}{\sigma \in \F S_n}\,$ and $\,\C S_n:=\,\vect[\F S_n]\,$.
	
	We define then $\:\displaystyle\C S:=\,\:\bigcup_{\mathclap{n\in \BB N}} \C S_n\:$, which we call the \emph{permutation algebra}.
}

Proposition \ref{open-represent} ensures that the $\,\C S_n\,$ and $\,\C S\,$ are $\ast$-algebras.


\section{Words and Observations}\label{sec_words}
%	\subsection{Representation of words}
	The representation of words over an alphabet in the unification algebra directly comes from the translation of church lists in linear logic and their interpretation in Geometry of Interaction models~\cite{girard_geometry_1989,girard_geometry_1995}. However, no knowledge of these topics is strictly needed and we can give a direct presentation of the notion.

\medskip
From now on, we fix a set of two disinguished constant symbols $\,\IO:=\{\:\i,\o\:\}\,$.


\define[word algebra]
	{To a finite set $\,\Sigma\,$ of constant symbols, we associate the $\ast$-algebra
	$$[\Sigma]:= \vect\set{\TT c\flow \TT d}{\TT c,\TT d\in\Sigma}
	$$
	When it is clear from the context, we will simply write $\,\Sigma\,$ in place of $\,[\Sigma]\,$.
	
	\smallskip
	The word algebra associated to $\,\Sigma\,$ is the $\ast$-algebra defined as 
	$$\,\C W_\Sigma:=\BF 1\tensor\Sigma\tensor\IO\tensor(\closed)^1\,$$
	\vspace{-5mm}
}

The words we consider are cyclic, with a begin/end marker $\,\start\,$, a reserved constant symbol. for example the word $\,\TT{0010}\,$ is to be thought of as $\,\start\TT{0010}=\TT{01}\!\start\!\TT{00}=\TT0\!\start\!\TT{001}=\cdots\,$

We consider therefore from now on that the alphabet we consider contains the symbol $\,\start\,$.

\define[word representation]
	{\label{words}Let $\,W=\start\TT c_1\dots \TT c_n\,$ be a word over $\,\Sigma\,$ of length $\,n\,$.
	
	A \emph{representation} $\,W(t_0,t_1,\dots,t_n)\in \C W_\Sigma^\concrete\,$ of $\,W\,$ is an isometric wiring determined by the distinct closed terms $\,t_0,t_1,\dots,t_n\,$, defined as
$$
	\begin{array}{ccl}
		W(t_0,t_1,\dots,t_n) :=		& &x\p\start\p\i\p (t_0\p y) \sflow x\p\TT c_1\p\o\p (t_1\p y) \\
				&+& x\p\TT c_1\p\i\p (t_1\p y) \sflow x\p\TT c_2\p\o\p (t_2\p y) \\
				&+&\ \cdots\ \\
				&+& x\p\TT c_n\p\i\p (t_n\p y) \sflow x\p\start\p\o\p (t_0\p y)
	\end{array}
$$
}


%		w(t_0,t_1,\dots,t_n)W' :=		& &x\p\start\p\o\p (t_0\p y) \sflow x\p\TT c_1\p\i\p (t_1\p y) \\
%				&+& x\p\TT c_1\p\o\p (t_1\p y) \sflow x\p\TT c_2\p\i\p (t_2\p y) \\
%				&+&\ \cdots\ \\
%				&+& x\p\TT c_n\p\o\p (t_n\p y) \sflow x\p\start\p\i\p (t_0\p y)

	\emph{Observations} will stand for the machines accepting and refusing words. They  lie in a particular $\ast$-algebra based on the representation of permutations presented in section \ref{permutation}.

\define[observation algebra]
	{An \emph{observation} over a finite set of symbols $\,\Sigma\,$ is any element of $\,\C O_\Sigma^\concrete\,$ where $\,\C O_\Sigma:=\,\closed\tensor\Sigma\tensor\IO\tensor\C S\,$, \textit{i.e.} a \emph{finite} sum of flows of the form
	$$(s'\p\TT c' \p \TT d' \flow s\p\TT c\p\TT d)\tensor\sigma
	$$
with $\,\TT c,\TT c'\in\Sigma\,$, $\,\TT d,\TT d'\in \IO\,$ and $\,s,s'\,$ closed terms.

\smallskip
Moreover when an observation happens to be an isometric wiring, we will call it an \emph{isometric observation}.
}



	
\section{Normativity: Independance from Representations}\label{sec_normativity}
	We are going to define the way observations accept and reject words. Indeed there is a issue with the notion of representation: an observation is an element of $\,\ualg\,$ and can therefore only interact with \emph{representations} of a word, and there are many possible representation of the same word (in definition \ref{words}, different choices of closed terms lead to different representations).

Therefore one has to ensure that acceptation or rejection is independent of the representation, so that the construction makes sense.

\define[automorphism]
	{An \emph{automorphism} of a $\ast$-algebra $\,\C A\,$ is a linear application $\,\varphi\,:\: \C A\rightarrow\C A\,$ such that for all~$\,F,G\in\C A\,$
	
\vbox{	\begin{itemize}
		\item $\varphi(FG)=\varphi(F)\varphi(G)$
		\item $\varphi(F^\dagger)=\varphi(F)^\dagger$
		\item $\varphi$ is injective
	\end{itemize}}
	If there is a partial isometry $\,U\in \C A\,$ such that $\,\varphi(a)=u^\dagger au\,$ for all $\,a\,$, then we say that $\,\varphi\,$ is an \emph{inner} automorphism.\nolinebreak
}

\define[normative pair]
	{\label{normpair}A pair $\,(\C A,\C B)\,$ of $\ast$-algebras is a \emph{normative pair} whenever any automorphism $\,\varphi\,$ of $\,\C A\,$ can be extended into an automorphism $\,\overline\varphi\,$ of $\:\alg^\dagger(\C A \cup\C B)\:$ such that $\,\overline\varphi(F)=F\,$ for any $\,F\in \C B\subseteq\alg^\dagger(\C A \cup\C B)\,$.
}

The two following theorems set the basis for a notion of acceptation/rejection independent of the representation of a word.

\smallskip
%\notation{Given a finite set of closed terms $\,S\,$, we write $\,\displaystyle1_S:=\sum_{\TT s\in S} s \flow s\,$.}

\theorem[automorphic representations]
	{\label{automorphic}Any two representations $\,W(t_i),W(u_i)\,$ of a word $\,W\,$ over $\,\Sigma\,$ are automorphic: there exists an automorphism $\,\varphi\,$ of $\,(\closed)^1\,$ such that $$\,(\Id_{\,\BF 1\tensor\Sigma\tensor\IO}\tensor\varphi)\big(W(t_i)\big)=W(u_i)\,$$
}

\proof{Take as $\,\varphi\,$ the inner automorphism associated with the isometric wiring $$\displaystyle\sum_i t_i\p x \flow u_i\p x \:\in\, (\closed)^1$$
}

\theorem[nilpotency and normative pairs]
	{\label{normative}Let $\,(\C A,\C B)\,$ be a normative pair, $\,F\in \C A\,$, $\,G\in \C B\,$ and $\,\varphi\,$ an automorphism of $\,\C A\,$, then $\,GF\,$ is nilpotent if and only if $\,G\,\varphi(F)\,$ is nilpotent.
}

\proof{Let $\,\overline\varphi\,$ be the extension of $\,\varphi\,$ as in defintion \ref{normpair}. We have for all $\,n\geq 1\,$ that $\,(G\varphi(F))^n=(\overline\varphi(G)\overline\varphi(F))^n=(\overline\varphi(GF))^n=\overline\varphi((GF)^n)\,$.

By injectivity of $\,\overline\varphi\,$, $\,(\varphi(G)F)^n=0\,$ if and only if $\,(GF)^n=0\,$.
}

\corollary[independance]
{\label{indep}If $\,\big((\C \closed)^1,\C B\big)\,$ is a normative pair, $\,W\,$ a word over $\,\Sigma\,$ and $\,F\in \closed\tensor\Sigma\tensor\IO\tensor\C B\,$, then $\,FW(t_i)\,$ is nilpotent for one choice of $\,t_i\,$ if and only if it is nilpotent for all choices of $\,t_i\,$.}

%Now we can use the tools from section \ref{actions} to build a particular normative pair.

The word and observation algebras we introduced earlier can be shown to form a normative pair.

\theorem[a normative pair]
{For any $\,\Sigma\,$, $\,\big((\closed)^1,\C S\big)\,$ is a normative pair}

\proof[sketch]{By simple computations, the set $$\,\C A:=\vect\set{\sigma F}{\sigma \in \C S \text{ and }\, F\in (\closed)^\infty}\,$$ can be shown to be a $\ast$-algebra, so that $\,\alg^\dagger(\C S\cup(\closed)^1)=\alg^\dagger(\C S\cup(\closed)^\infty)=\C A\,$.

Now if $\,\varphi\,$ is an automorphism of $\,(\closed)^1\,$, it can be written as $\,\varphi(t\p x)=\psi(t)\p x\,$ with $\,\psi\,$ an automorphism of $\,\closed\,$. We define for $\,F=F_1\tensor\cdots\tensor F_n\tensor 1\in(\closed)^n\,$, $\,\tilde\phi(F):=\psi(F_1)\tensor\cdots\tensor \psi(F_n)\tensor 1\,$ which gives an automorphism of $\,(\closed)^\infty\,$ by linearity. Finally we extend $\,\tilde\varphi\,$ to $\,\C A\,$ by $\,\overline\varphi(\sigma F):=\sigma\tilde\varphi(F)\,$
}

\medskip
\remark{Here we sketched a direct proof for brevity, but this can also be shown by involving a little more mathematical structure (actions of permutations on the unbounded tensor and crossed products) which would give a more synthetic proof.}

\medskip
Therefore the notion of the language recognized by an observation makes sense, \textit{via} corollary \ref{indep}.

\define[language of an observation]
	{Let $\,\phi\in\C O_\Sigma\,$ be an observation over $\,\Sigma\,$. The \emph{language recognized by $\,\phi\,$} is the following set of words over $\,\Sigma\,$
	
	$$\lang(\phi):=\set{W\:\text{ word over }\,\Sigma}{\phi\,W(t_i)\,\text{ nilpotent for any choice of }\,(t_i)} $$
	}


\section{Wirings and Logarithmic Space}\label{sec_logspace}
	Now that we have defined our framework and showed how observations can compute, we study the complexity of deciding whenever an observation accepts a word (\ref{subsec_soundness}), and how wirings can decide any langage in \textsc{(N)Logspace} (\ref{subsec_completness}).
	\subsection{Soundness of Observations}
	\label{subsec_soundness}
	The aim of this subsection is to prove the following theorem:

% Compacte
\vbox{
\theorem[space soundness]
	{\label{soundness}Let $\,\phi\in\C O_\Sigma^+\,$ be an observation over $\,\Sigma\,$.
	\begin{itemize}
		\item $\lang(\phi)\,$ is decidable in non-deterministic logarithmic space.
		\item If $\,\phi\,$ is isometric, then $\lang(\phi)\,$ is decidable in deterministic logarithmic space.
	\end{itemize}
}
}
Actually, the result stands for the complements of these languages, but as {\sc co-NLogspace = NLogspace} by the Immerman-Szelepcsényi theorem, this makes no difference.

\smallskip
The main tool for this purpose is the notion of \emph{computation space}: a finite dimensional subspace of $\,\closedh\,$ on which we will be able to observe the behavior of our wirings. It can be understood as the place where all the relevant computation takes place.% This notion is reminiscent of the notion of separating vector 

\define[separating space]
{A subspace $\,E\,$ of $\,\closedh\,$ is \emph{separating} for a wiring $\,F\,$ whenever $\,F(E)\subseteq E\,$ and $\,F^n(E)=0\,$ implies $\,F^n=0\,$.}

Observations are \emph{finite} sums of wirings. We can naturally associate a finite-dimensional vector space to an observation and a finite set of closed terms.

\define[computation space]
{\label{compspace}Let $\,\{\,t_0,\dots, t_n\,\}\,$ be a set of distinct closed terms and $\,\phi\in\C O_\Sigma^\concrete\,$ an observation.

Let $\,N(\phi)\,$ be the smallest integer and $\,\TT S(\phi)\,$ the smallest (finite) set of closed terms such that $\,\phi \in (\TT S(\phi)^\ast\ptensor \Sigma^\ast\tensor\IO^\ast)\tensor\C S_{N(\phi)}\,$.

\smallskip
The \emph{computation space} $\,\comp_\phi (t_0\dots t_n)\,$ is the subspace of $\,\closedh\,$ generated by the terms
\[s\p \TT c\p \TT d \p (\,a_1\p\,\cdots\,\p a_{N(\phi)}\,\p\, \start)\]
where $\,s\in \TT S(\phi)\,$, $\,\TT c\in \Sigma\,$, $\,\TT d\in\IO\,$ and the $\,a_i\in \{\,t_0,\dots, t_n\,\}\,$.

The dimension of $\,\comp_\phi (t_0\dots t_n)\,$ is $\,|\Sigma| 2 (n+1)^{N(\phi)}|\TT S(\phi)|\,$ (where $\,|A|\,$ is the cardinal of $\,A\,$), which is polynomial in $\,n\,$. 
}



\lemma[separation]
{\label{sep}For any observation $\,\phi\,$ and any word $\,W\,$, the space $\,\comp_\phi(t_0\dots t_n)$ is separating for the wiring $\,\phi \,W(t_0\dots t_n)\,$.}

\proof[of Theorem~\ref{soundness}]{With these lemmas at hand, we can define the non-deterministic algorithm below. It takes as an input the representation $\,W(t_0\dots t_n)\,$ of a word $\,W\,$ of length $\,n\,$. $\phi\,$. 

Being a constant, one can compute once and for all $\,N(\phi)\,$ and $\,S(\phi)\,$.

\begin{multicols}{2}
%\begin{wrapfigure}[18]{l}{0.5\textwidth}
\begin{algorithmic}[1]
%\STATE compute $\,N(\phi)\,$ and $\,S(\phi)\,$
\STATE $D\gets 2|\TT S(\phi)|\,|\Sigma|(n+1)^{N(\phi)}$
\STATE $C\gets 0$
\STATE pick a term $\,v\in\comp_\phi(t_i)\,$\label{pick1}
\WHILE{$C\leq D$}
	\IF{$(\phi W(t_i))(v)=0$} \label{if}
		\RETURN ACCEPT
	\ENDIF
	\STATE pick a term $\,v'\,$ in $\,(\phi W(t_i))(v)\,$ \label{pick2}
	\STATE $v\gets v'$
	\STATE $C\gets C+1$
\ENDWHILE
\RETURN REJECT
\end{algorithmic}
\end{multicols}
%\caption{Nilpotency algorithm}
%\end{wrapfigure}

\smallskip
All computation paths (the \enquote{pick} at lines \ref{pick1} and \ref{pick2} being non-deterministic choices) accept if and only if $\,(\phi W(t_i))^n(\comp_\phi(t_i))=0\,$ for some $\,n\,$ lesser or equal to the dimension $D$ of the computation space $\,\comp_\phi(t_i)\,$.
By Lemma \ref{sep}, this is equivalent to $\,\phi W(t_i)\,$ being nilpotent.

The vector chosen at lines \ref{pick1} is representable by an integer of size at most $D$ and is erased by the one chosen at line \ref{pick2} every time we go through the {\bf while}-loop.
$C$ and $D$ are integers proportional to the dimension of the computation space, which is representable in logarithmic space in the size of the input (Definition~\ref{compspace}).
% Compacte

\smallskip
%The algorithm can run in logarithmic space, because it only needs to store a vector of a vector space of dimension $\,D=2|\TT S(\phi)|\,|\Sigma|(n+1)^{N(\phi)}\,$ (which can be represented as an integer lesser or equal to $\,D\,$), and the two integers $\,C,D\,$.
The computation of $\,(\phi W(t_i))(v)\,$ at line \ref{if} and \ref{pick2} and can be performed in logarithmic space by Proposition \ref{unif-logspace}, as we are unifying closed terms with linear terms.
%all terms used in words representations and observations are linear and share no variables.
% Compacte

\smallskip
Moreover, if $\,\phi\,$ is an isometric wiring, $\,(\phi W(t_i))(v)\,$ consists of a single vector instead of a sum by Lemma \ref{lem_isom}, and there is therefore no non-deterministic choice to be made at line \ref{pick2}.
It is then enough to run the algorithm enumerating all possible vectors of $\,\comp_\phi(t_i)\,$ at line \ref{pick1} to determine the nilpotency of $\,\phi W(t_i)\,$.
}

	\subsection{Completeness: Representing Pointer Machines as Wirings}
	\label{subsec_completness}
	To prove the converse of Theorem \ref{soundness}, we prove that wirings can encode a special kind of read-only multi-heads Turing Machines: pointers machines.
The definition of this model will be guided by our understanding of the computation represented in wirings.
It will scan the input without the ability to store anything else than the \emph{positions} of its \emph{pointers} and the current state.
%Acceptation will be defined as the opposite of looping, and it is natural to see the course of the computation as universally non-deterministic.

Theses specificities do not enhance nor lower the computational power of our device, they just need to be properly tamed, to get back to a model close to Multi-Head Finite Automata and be able to decide any {\sc Logspace}-language.
For a survey of this topic, one may consult \cite[Chap.4]{Aubert2013b}, the main novelty of this work is to prove that deterministic computation is represented by reversible operators.

%This model of computation can naturally be conceived as read-only multi-heads Turing Machines, or as Multi-Head Finite Automata: our device will scan the input without the ability to store anything else than the \emph{positions} of its \emph{pointers} and the current state.
%We will slightly modify this classical device, to ease the encoding as wirings: only one pointer will have the ability to move on the input, and a second step of computation will allow to sitch the position of two pointers.
%This fine-grained way of representating the transitions of our device helps to see how wirings compute.
%Apart from the result, this also provides a way of seeing how wiring are computing.

%Pointer machines \cite{ben-amram_what_1995} are theoretical devices that aim at modelling computation in a way that varies from turing machines: the data is a read-only sequence of symbols, and there is no memory tape. Instead, the machine manipulates pointers to various point of the input.

\define[pointer machine]
{A pointer machine over an alphabet $\,\Sigma\,$ is a tuple $\,(N,\TT S,\Delta)\,$ where
\begin{itemize}
	\item $N\neq 0\,$ is an integer, the \emph{number of pointers},
	\item $\TT S\,$ is a finite set, the \emph{states} of the machine,
	\item $\Delta \,\subseteq\, (\TT S\times\Sigma\times\IO)\times(\TT S\times\Sigma\times\IO)\times \F S_N\:$, the \emph{transitions} of the machine
	
	(we will write $\,(s,\TT c,\TT d) \rightarrow (s',\TT c',\TT d') \times \sigma\,$ the transitions, for readability).
\end{itemize}
A pointer machine will be called \emph{deterministic} if for any $\,A \,\in\, \TT S\times \Sigma\times \IO\,$, there is at most one $\,B\,\in\, \TT S\times \Sigma\times \IO\,$ and one $\,\sigma\in \F S_n\,$ such that $\,A\rightarrow B \times \sigma\,\in\,\Delta\,$.
In that case we can see $\,\Delta\,$ as a partial function, and we say that $\,M\,$ is \emph{reversible} if $\,\Delta\,$ is a partial injection.
}
We call the first of the $\,N\,$ pointers the \emph{main} pointer, it is the only one that can read the input.
The other pointers are refered to as the \emph{auxiliairy} pointers.
An auxiliary pointer will be able to become the main pointer along the computation thanks to permutations.

\define[configuration]
{Given the length $\,n\,$ of a word $\,W=\start\TT c_1\dots \TT c_n\,$ over $\,\Sigma\,$ and a pointer machine $\,M=(N,\TT S,\Delta)\,$, a \emph{configuration} of $\,(M,n)\,$ is an element of 
\[\,\TT S\times\Sigma\times\IO\times\{0,1,\dots,n\}^N\,\]}

As usual, the element of $\,\TT S\,$ is the state of the machine and the element of $\,\Sigma\,$ is the letter the main pointer points at.
The element of $\,\IO\,$ is the direction of the next move of the main pointer, and the elements of $\,\{0,1,\dots,n\}^N\,$ correspond to the positions of the (main and auxiliary)  pointers on the input.

\smallskip

As the input tape is considered cyclic with a special symbol marking the beginning of the word (recall Definition \ref{words}), the pointer positions are integers \emph{modulo} $\,n+1\,$ for an input word of length $\,n\,$.

\define[transition]
{Let $\,W\,$ be a word and $\,M=(N,\TT S,\Delta)\,$ be a pointer machine.
A \emph{transition} of $\,M\,$ on input $\,W\,$ is a triple of configurations
\[s,\TT c,\TT d,(p_1,\dots,p_N) \trans{\TT{MOVE}}{} s,\TT c',\overline{\TT d},(p_1',\dots,p_N') \trans{\TT{SWAP}}{} s',\TT c'',\TT d',(p_{\sigma(1)}',\dots,p_{\sigma(N)}') \]
such that
\begin{enumerate}
	\item if $\,\TT d\in\IO\,$, $\,\overline{\TT d}\,$ is the other element of $\,\IO\,$,
	\item $p_1'=p_1+1\,$ if $\,\TT d=\i\,$ and $\,p_1'=p_1-1\,$ if $\,\TT d=\o\,$,
	\item $p_i'=p_i\,$ for $\,i\neq 1\,$,
	\item $\TT c\,$ is the letter at position $\,p_1\,$ and $\,\TT c'\,$ is the letter at position $\,p_1'\,$, \label{condition}
	\item and $(s,\TT c',\overline{\TT d}) \rightarrow (s',\TT c'',\TT d') \times \sigma\,$ belongs to $\,\Delta\,$.
\end{enumerate}
}
There is no constraint on $\,c''\,$, but every time this value differs from the letter pointed by $\,p_{\sigma(1)}'\,$, the computation will halt on the next \texttt{MOVE} phase, because there is a mismatch between the value that is supposed to have been read and the actual bit of $\,W\,$ stored at this position, and that would contradict the first part of item \ref{condition}.
In terms of observations, the \texttt{MOVE} phase corresponds to the application of the word, whereas the \texttt{SWAP} phase corresponds to the application of the observation.

%Note that the maintenance of $\,\TT c''\,$ as the letter pointed to by the (new) main pointer is not required at the \texttt{SWAP} phase, which will cause most computation to stop unexpectedly.
%The point is that it is possible to implement that maintainance using the states of the machine (typically, add a store for the values of the pointers, which has finitely many configurations, to the state of the machine), and therefore we favor the simpler and more liberal definition.

\define[acceptation]
{\label{translate}We say that $\,M\,	$ accepts $\,W\,$ if any sequence of transitions $\,\big(C_i\trans{\TT{MOVE}}{}C_i'\trans{\TT{SWAP}}{}C_i''\big)\,$ such that $\,C''_i=C_{i+1}\,$ for all $\,i\,$ is necessarily finite.

We write $\,\lang(M)\,$ the set of words accepted by $\,M\,$.}

This means informally that we consider that a pointer machine accepts a word if it cannot ever loop, from whatever configuration it starts from.
To retrieve computation as one could expect it, it is necessary to write programs with a lot of guesses on the values to be read, to store some information in the configurations and to keep in mind that the number of configurations can always be bounded.
That a lot of path of computations accepts \enquote{wrongly} is no worry, since only rejection is meaningful: our pointer machines compute in a \enquote{universally non-deterministic} way to stick to the acceptance condition of wirings, nilpotency.

%Of course, this rather unusual acceptance condition is given with the acceptance condition of wirings in mind: nilpotency.
%As a side-effect, it gives evidences that our pointer machines compute in a \enquote{universally non-deterministic} way.

%This computational behaviour is suited to be embeded into the wirings, it reflects their wild way of computing, massively non-deterministic: 

\theorem[space completeness]
{\label{pointerl}If $\,L\in \text{\sc NLogspace}\,$, then there exist a pointer machine $\,M\,$ such that $\,\lang(M)=L\,$.
Moreover, if $\,L\in \text{\sc Logspace}\,$ then $\,M\,$ can be chosen to be reversible.
}
\proof[sketch]{%
It is well-known that (non-)Deterministic Multi-Head Finite Automata characterize {\sc (N)Logspace} \cite{Hartmanis1972}.
It takes little effort to see that our pointer machines are just a reasonnable rearrangement of Multi-Head Finite Automata: that only one pointer may move does not lower the computational power of our device since it is always possible to switch the rôle of the pointers.
We can get back to the expected notion of transition by encoding into the states the values that can be read by the pointers.

That acceptation and rejections are \enquote{reversed} is harmless in the deterministic (or equivalently reversible \cite{Lange2000}) case, and uses that {\sc co-NLogspace = NLogspace} to get the expected result in the non-deterministic case.
}

As we said, our pointer machines are designed to be easily simulated by wirings, so that we get the expected result almost for free.

\theorem[simulation]
{If $\,L\in \text{\sc NLogspace}\,$, then there exist an observation $\,\phi\in\C O_\Sigma\,$ such that $\,\lang(\phi)=L\,$.
Moreover, if $\,L\in \text{\sc Logspace}\,$ then $\,\phi\,$ is an isometric wiring.
}

\proof{%
By Theorem \ref{pointerl}, there exists a pointer machine $\,M=(N,\TT S,\Delta)\,$ such that $\lang(M) = L$.
We associate to $\,S\,$ a set of distinct closed terms $\,[\TT S]\,$ and write $\,[s]\,$ the term associated to $\,s\,$.
To any element $\,D=(s,\TT c,\TT d) \rightarrow (s',\TT c',\TT d') \times \sigma\,$ of $\,\Delta\,$ we associate the flow 
\[[D]:=([s']\p\TT c'\p\TT d' \flow [s]\p\TT c\p\TT d) \tensor \sigma \:\in[\TT S]\tensor\Sigma\tensor\IO\tensor\C S_n\,\]
and we define the wiring $\,[M]\in\C O_\Sigma^+\,$ as $\,\displaystyle\sum_{D\in \Delta} [D]\,$.

One can easily check that this translation preserves the language recognized and relates reversibility with isometricity: in fact, $\,M\,$ is reversible if and only if $\,[M]\,$ is an isometric wiring.
So if $\,L\in \text{\sc Logspace}\,$, $\,M\,$ is deterministic and can always be chosed to be reversible.
}

\section*{Discussion}
 \addcontentsline{toc}{section}{Discussion}
With respect to the earlier \cite{girard_normativity_2012,aubert_characterizing_2012,seiller_logarithmic_2013},
this work begins to clarify one point: what is really needed to carry on the construction that captures logarithmic space computation?

Indeed, these earlier works used in place of the unification algebra the so-called \emph{hyperfinite factor},
which involved advanced notions of von~Neumann algebras theory.

Our work shows that the von Neumann structure is indeed not indispensible,
whereas the ability to represent the action of permutation groups on an unbounded tensor product seems to be an important element in the relation to pointer machines.

\smallskip
The language the unification algebra gives us a twofold point of view on computation, either through algebraic structures or pointer machines. We can therefore start exploring possible variations of the construction, combining intuitions from both points of view.

For instance, a natural operation one would like to define on pointer machines would be the one the resets the main pointer to the initial position holding the symbol $\,\star\,$.
This is not possible within the setting of this article, because of the notion of normative pair:
this would require to know in advance the term $\,t_0\,$ corresponding to the initial position.

\smallskip
This indicates that the choice of a normative pair can affect the expressivity of the construction.
As a general principle, one has that the more restrive the notion of representation of word is,
the more liberal that of observation can become.
Whether and how this can affect the corresponding complexity class is definitely a direction for future work.


%\addcontentsline{toc}{section}{Bibliography}
\bibliographystyle{splncs}
\bibliography{girard,complexity,phd,unification,goi}
\addcontentsline{toc}{section}{References}

