\textbf{Proof theory and complexity theory.} Since the introduction of so-called light logics \cite{girard_light_1994}, several authors contributed to the topic of implicit complexity theory using tools coming from proof theory and more specifically linear logic \cite{girard_linear_1987}.

The study of geometrical properties of proofnets \cite{girard_linear_1987}, in terms for instance of stratitification of exponential boxes \cite{baillot_linear_2010} led to a clearer understanding of the time complexity of the cut-elimination procedure. Space complexity has also been studied from this perspective \cite{schopp_stratified_2007,gaboardi_logical_2008}.

\smallskip\noindent
\textbf{Geometry of Interaction.} As the study of cut-elimination has grown as a central topic in proof theory, its mathematical modelling became of great interest. The Geometry of Interaction \cite{girard_towards_1989} research program led to models of cut-elimination in terms of paths in proofnets \cite{asperti_paths_1994}, token machines \cite{laurent_token_2001} and operator algebras \cite{girard_geometry_1989}.
The relationship of these ideas with complexity theory has also been investigated \cite{schopp_space-efficient_2006,baillot_elementary_2001}.

The tools shaped by this approach have also been used to study directly complexity theory from an algebraic point of view \cite{girard_normativity_2012,aubert_characterizing_2012,seiller_logarithmic_2013}. These works are the main source of inspiration for this article.

\smallskip\noindent
\textbf{Unification.} The unification technique has been used \cite{girard_geometry_1995,baillot_elementary_2001,girard_three_lightings} to provide a setting where one can model the cut-elimination procedure in a finitary way: first-order terms with variables offer a way to manipulate infinite sets (of instances of terms) with a finite language.

It turns out that this language is expressive enough to encode various algebraic structures, including a notion of unbounded tensor product and a representation of finite permutation groups. These were provided by the use of the theory of von Neumann algebras \cite{girard_normativity_2012,aubert_characterizing_2012,seiller_logarithmic_2013}.

\smallskip\noindent
\textbf{Pointer machines.} The study of space efficient computations naturally leads to consider pointer machines as a mathematical model of computing.%\cite{hofmann_pointer_2009}.%,hofmann_pure_2010}.

How operators simulate computation has been invented recently \cite{girard_normativity_2012}, and studied more precisely \cite{aubert_characterizing_2012,seiller_logarithmic_2013} thanks to a read-only variant of the Turing Machines, a natural characterization of {\sc LogSpace} computation.
%In \cite{girard_normativity_2012}, how operators simulate computation is briefly explained in terms of pointer machines, an idea that has been made precise by the later \cite{aubert_characterizing_2012,seiller_logarithmic_2013}.


%We introduce here a notion of pointer machine that is specifically designed to be easily translated into 

\bigskip\noindent
\textbf{Contribution.} We carry on the methodology of bridging Geometry of Interaction and pointer machines  \cite{girard_normativity_2012,aubert_characterizing_2012,seiller_logarithmic_2013}, but we give a simpler presentation based on unification that does not rely on the theory of von Neumann algebras.
We show that in the deterministic case, reversibility (of machines) is related to the algebraic notion of isometricity (of observations), an insight of the core of the relation that was absent in the aforementionned works.

%\bigskip\noindent
%\textbf{Contribution.} We carry on the methodology of \cite{girard_normativity_2012,aubert_characterizing_2012,seiller_logarithmic_2013}, but we give a simpler presentation based on unification that does not rely on the theory of von Neumann algebras. 
%We relate reversibility (of machines) and the algebraic notion of isometricity (of observations), a characterisation that was absent in the aforementionned works.

\medskip\noindent
\textbf{Organisation of this article.} In Sect.\ref{sec_unification} we review some classical results on unification of first-order terms and use them to build the algebra that will constitute our computationnal setting.

We explain in Sect.\ref{sec_words} how words and computing devices (observations) can be modeled by particular elements of this algebra.
The way they interact to yield a notion of language recognized by an observation is described in Sect.\ref{sec_normativity}.

Finally, we show in Sect.\ref{sec_logspace} that our construction captures exactly logarithmic space computation, both deterministic and non-deterministic.
