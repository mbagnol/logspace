With respect to earlier \cite{girard_normativity_2012,aubert_characterizing_2012,seiller_logarithmic_2013} this work start to clarify one point: what is really needed to carry on the construction that captures logarithmic space computation?

Indeed, earlier work used in place of the unification algebra the so-called \emph{hyperfinite factor}, which involved advanced notions of von~Neumann algebras theory. Our work shows that the von Neumann structure is indeed not indispensible, whereas the ability to represent the action of permutation groups on an unbounded tensor product seems to be an important point in the relation to pointer machines.

\smallskip
The language of unification gives us a twofold point of view on computation: now that we related wirings and pointer machines, we can start exploring possible extension of our construction, keeping some mathematical structuring in mind.

For instance, a natural operation one would like to define on pointer machines would be the one the resets the main pointer to the initial position holding the symbol $\,\star\,$. This is not possible within the setting of this article, because of the notion of normative pair: this would require to know in advance the term $\,t_0\,$ corresponding to the initial position.


