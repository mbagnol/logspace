The representation of words over an alphabet in the unification algebra directly comes from the translation of church lists in linear logic and their interpretation in Geometry of Interaction models~\cite{girard_geometry_1989,girard_geometry_1995}. However, no knowledge of these topics is strictly needed and we can give a direct presentation of the notion.

\medskip
From now on, we fix a set of two disinguished constant symbols $\,\IO:=\{\:\o,\i\:\}\,$.


\define[word algebra]
	{To a finite set $\,\Sigma\,$ of constant symbols, we associate the $\ast$-algebra
	$$[\Sigma]:= \vect\set{\TT c\flow \TT d}{\TT c,\TT d\in\Sigma}
	$$
	When it is clear from the context, we will simply write $\,\Sigma\,$ in place of $\,[\Sigma]\,$.
	
	\smallskip
	The \emph{word algebra} associated to $\,\Sigma\,$ is the $\ast$-algebra defined as 
	$$\,\C W_\Sigma:=\BF 1\tensor\Sigma\tensor\IO\tensor(\closed)^1\,$$
	\vspace{-8mm}
}

The words we consider are cyclic, with a begin/end marker $\,\start\,$, a reserved constant symbol. for example the word $\,\TT{0010}\,$ is to be thought of as $\,\start\TT{0010}=\TT{01}\!\start\!\TT{00}=\TT0\!\start\!\TT{001}=\cdots\,$

We consider therefore from now on that the alphabet $\,\Sigma\,$ always contains the symbol $\,\start\,$.

\define[word representation]
	{\label{words}Let $\,W=\start\TT c_1\dots \TT c_n\,$ be a word over $\,\Sigma\,$.
	
	A \emph{representation} $\,W(t_0,t_1,\dots,t_n)\in \C W_\Sigma^\concrete\,$ of $\,W\,$ is an isometric wiring determined by the distinct closed terms $\,t_0,t_1,\dots,t_n\,$, defined as
$$
	\begin{array}{ccl}
		W(t_0,t_1,\dots,t_n) :=		& &x\p\start\p\i\p (t_0\p y) \sflow x\p\TT c_1\p\o\p (t_1\p y) \\
				&+& x\p\TT c_1\p\i\p (t_1\p y) \sflow x\p\TT c_2\p\o\p (t_2\p y) \\
				&+&\ \cdots\ \\
				&+& x\p\TT c_n\p\i\p (t_n\p y) \sflow x\p\start\p\o\p (t_0\p y)
	\end{array}
$$
\vspace{-5mm}
}

