To prove the converse of Theorem \ref{soundness}, we prove that wirings can encode a special kind of Turing Machines\footnote{%
To encode Turing Machines in unification and to relates computation steps to steps of unification was already done for instance in \cite{Itai1987}, but this latter was about calculability, and did not really took into account complexity.}%
: pointers machines.
This model of computation can naturally be conceived as read-only multi-heads Turing Machines, or as Multi-Head Finite Automata: our device will scan the input without the ability to store anything else than the \emph{positions} of its \emph{pointers} and the current state.
For a survey of this topic, one may consult \cite[Chap.4]{Aubert2013b}.

We will slightly modify this classical device, to ease the encoding as wirings: only one pointer will have the ability to move on the input, and a second step of computation will allow to sitch the position of two pointers.
This fine-grained way of representating the transitions of our device helps to see how wirings compute.
%Apart from the result, this also provides a way of seeing how wiring are computing.

%Pointer machines \cite{ben-amram_what_1995} are theoretical devices that aim at modelling computation in a way that varies from turing machines: the data is a read-only sequence of symbols, and there is no memory tape. Instead, the machine manipulates pointers to various point of the input.

\define[pointer machine]
{A pointer machine over an alphabet $\,\Sigma\,$ is a tuple $\,(N,\TT S,\Delta)\,$ where
\begin{itemize}
	\item $N\neq 0\,$ is an integer, the \emph{number of pointers},
	\item $\TT S\,$ is a finite set, the \emph{states} of the machine,
	\item $\Delta \,\subseteq\, (\TT S\times\Sigma\times\IO)\times(\TT S\times\Sigma\times\IO)\times \F S_N\:$, the \emph{transitions} of the machine
	
	(we will write $\,(s,\TT c,\TT d) \rightarrow (s',\TT c',\TT d') \times \sigma\,$ the transitions, for readability).
\end{itemize}
A pointer machine will be called \emph{deterministic} if for any $\,A \,\in\, \TT S\times \Sigma\times \IO\,$, there is at most one $\,B\,\in\, \TT S\times \Sigma\times \IO\,$ and one $\,\sigma\in \F S_n\,$ such that $\,A\rightarrow B \times \sigma\,\in\,\Delta\,$.
In that case we can see $\,\Delta\,$ as a partial function, and we say that $\,M\,$ is \emph{reversible} if $\,\Delta\,$ is a partial injection.
}
We call the first of the $\,N\,$ pointers the \emph{main} pointer, it is the only one that can read the input.
The other pointers are refered to as the \emph{auxiliairy} pointers.
An auxiliary pointer will be able to become the main pointer thanks to the permutations.

\define[configuration]
{Given the length $\,n\,$ of a word $\,W=\start\TT c_1\dots \TT c_n\,$ over $\,\Sigma\,$ and a pointer machine $\,M=(N,\TT S,\Delta)\,$, a \emph{configuration} of $\,(M,n)\,$ is an element of 
\[\,\TT S\times\Sigma\times\IO\times\{0,1,\dots,n\}^N\,\]}

As usual, the element of $\,\TT S\,$ is the state the machine is in and the element of $\,\Sigma\,$ is the letter the main pointer points at.
The element of $\,\IO\,$ is the direction of the next move of the main pointer, and the element of $\,\{0,1,\dots,n\}^N\,$ correspond to the positions of the pointers on the input.

\smallskip

As the input tape is considered cyclic with a special symbol marking the beginning of the word (recall Definition \ref{words}), the pointer positions are integers \emph{modulo} $\,n+1\,$ for an input word of length $\,n\,$.

\define[transition]
{Let $\,W\,$ be a word and $\,M=(N,\TT S,\Delta)\,$ be a pointer machine.
We define $\,M(W)\,$ to be $\,M\,$ with all its pointers on $\,\star\,$, $\,W\,$ written on its input state and in some pre-defined \emph{inital state}.
A \emph{transition} of $\,M(W)\,$ is a triple of configurations
\[s,\TT c,\TT d,(p_1,\dots,p_N) \trans{\TT{MOVE}}{} s,\TT c',\overline{\TT d},(p_1',\dots,p_N') \trans{\TT{SWAP}}{} s',\TT c'',\TT d',(p_{\sigma(1)}',\dots,p_{\sigma(N)}') \]
such that
\begin{enumerate}
	\item if $\,\TT d\in\IO\,$, $\,\overline{\TT d}\,$ is the other element of $\,\IO\,$,
	\item $p_1'=p_1+1\,$ if $\,\TT d=\i\,$ and $\,p_1'=p_1-1\,$ if $\,\TT d=\o\,$,
	\item $p_i'=p_i\,$ for $\,i\neq 1\,$,
	\item $\TT c\,$ is the letter at position $\,p_1\,$ and $\,\TT c'\,$ is the letter at position $\,p_1'\,$, \label{condition}
	\item and $(s,\TT c',\overline{\TT d}) \rightarrow (s',\TT c'',\TT d') \times \sigma\,$ belongs to $\,\Delta\,$.
\end{enumerate}
}
There is no constraint on $\,c''\,$, but every time this value differs from the letter pointed by $\,p_{\sigma(1)}'\,$, the computation will halt on the next \texttt{MOVE} phase, because there is a mismatch between the value that is supposed to have been read and the actual bit of $\,W\,$ stored at this position.
At item \ref{condition}
This computational behaviour is suited to be embeded into the wirings, it reflects their wild way of computing, massively non-deterministic.

%Note that the maintenance of $\,\TT c''\,$ as the letter pointed to by the (new) main pointer is not required at the \texttt{SWAP} phase, which will cause most computation to stop unexpectedly.
%The point is that it is possible to implement that maintainance using the states of the machine (typically, add a store for the values of the pointers, which has finitely many configurations, to the state of the machine), and therefore we favor the simpler and more liberal definition.

\define[acceptation]
{\label{translate}We say that a pointer machine $\,M\,$ accepts the word $\,W\,$ if any sequence of transitions $\,\big(C_i\trans{\TT{MOVE}}{}C_i'\trans{\TT{SWAP}}{}C_i''\big)\,$ of $\,M(W)\,$ such that $\,C''_i=C_{i+1}\,$ for all $\,i\,$ is necessarily finite.

We write $\,\lang(M)\,$ the set of words accepted by $\,M\,$.}

This means informally that we consider that a pointer machine accepts a word if it cannot ever loop, from whatever configuration it starts from.
Of course, this rather unusual acceptance condition is given with the acceptance condition of wirings in mind: nilpotency.
As a side-effect, it gives evidences that our pointer machines compute in a \enquote{universally non-deterministic} way.


\theorem[space completeness]
{\label{pointerl}If $\,L\in \text{\sc NLogspace}\,$, then there exist a pointer machine $\,M\,$ such that $\,\lang(M)=L\,$.
Moreover, if $\,L\in \text{\sc Logspace}\,$ then $\,M\,$ can be chosen to be reversible.
}
\proof[sketch]{%
It is well-known that (non-)Deterministic Multi-Head Finite Automata characterize {\sc (N)Logspace} \cite{Hartmanis1972}.
It takes little effort to see that pointer machines are just a reasonnable rearrangement of Multi-Head Finite Automata: that only one pointer may move does not lower the computational power of our device since it is always possible to switch the rôle of the pointers.
We can get back to the expected notion of transition by encoding into the states the values that can be read by the pointers.
That acceptation and rejections are \enquote{reversed} is harmless in the deterministic (reversible) case, and uses that {\sc co-NLogspace = NLogspace} \cite{Immerman1988} to get the expected result in the non-deterministic case.
}

As we said, our pointer machines are designed to be easily simulated by wirings, so that we get the expected result almost for free.

\theorem[simulation]
{If $\,L\in \text{\sc NLogspace}\,$, then there exist an observation $\,\phi\in\C O_\Sigma\,$ such that $\,\lang(\phi)=L\,$.
Moreover, if $\,L\in \text{\sc Logspace}\,$ then $\,\phi\,$ is an isometric wiring.
}

\proof{%
By Theorem \ref{pointerl}, there exists a pointer machine $\,M=(N,\TT S,\Delta)\,$ such that $\lang(M) = L$.
We associate to $\,S\,$ a set of distinct closed terms $\,[\TT S]\,$ and write $\,[s]\,$ the term associated to $\,s\,$.
To any element $\,D=(s,\TT c,\TT d) \rightarrow (s',\TT c',\TT d') \times \sigma\,$ of $\,\Delta\,$ we associate the flow 
\[[D]:=([s']\p\TT c'\p\TT d' \flow [s]\p\TT c\p\TT d) \tensor \sigma \:\in[\TT S]\tensor\Sigma\tensor\IO\tensor\C S_n\,\]
and we define the wiring $\,[M]\in\C O_\Sigma^+\,$ as $\,\displaystyle\sum_{D\in \Delta} [D]\,$.

One can easily check that this translation preserves the language recognized and relates reversibility with isometricity: in fact, $\,M\,$ is reversible if and only if $\,[M]\,$ is an isometric wiring.
So if $\,L\in \text{\sc Logspace}\,$, $\,M\,$ is deterministic and can always be chosed to be reversible.
}