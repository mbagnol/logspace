To prove the converse of Theorem \ref{soundness}, we prove that wirings can encode a special kind of read-only multi-heads Turing Machines: pointers machines.
The definition of this model will be guided by our understanding of the computation of wirings: they won't have the ability to write and acceptation will be defined as termination of all paths of computation.
%It will scan the input without the ability to store anything else than the \emph{positions} of its \emph{pointers} and the current state.
%Acceptation will be defined as the opposite of looping, and it is natural to see the course of the computation as universally non-deterministic.
%Theses specificities do not enhance nor lower the computational power of our device, they just need to be properly tamed, to get back to a model close to Multi-Head Finite Automata and be able to decide any {\sc Logspace}-language.
For a survey of this topic, one may consult the first author's thesis \cite[Chap.4]{Aubert2013b}, the main novelty of this work is to prove that reversible computation is represented by isometric operators.

%This model of computation can naturally be conceived as read-only multi-heads Turing Machines, or as Multi-Head Finite Automata: our device will scan the input without the ability to store anything else than the \emph{positions} of its \emph{pointers} and the current state.
%We will slightly modify this classical device, to ease the encoding as wirings: only one pointer will have the ability to move on the input, and a second step of computation will allow to sitch the position of two pointers.
%This fine-grained way of representating the transitions of our device helps to see how wirings compute.
%Apart from the result, this also provides a way of seeing how wiring are computing.

%Pointer machines \cite{ben-amram_what_1995} are theoretical devices that aim at modelling computation in a way that varies from turing machines: the data is a read-only sequence of symbols, and there is no memory tape. Instead, the machine manipulates pointers to various point of the input.

\define[pointer machine]
{A \emph{pointer machine} over an alphabet $\,\Sigma\,$ is a tuple $\,(N,\TT S,\Delta)\,$ where
\begin{itemize}
	\item $N\neq 0\,$ is an integer, the \emph{number of pointers},
	\item $\TT S\,$ is a finite set, the \emph{states} of the machine,
	\item $\Delta \,\subseteq\, (\TT S\times\Sigma\times\IO)\times(\TT S\times\Sigma\times\IO)\times \F S_N\:$, the \emph{transitions} of the machine
	
	(we will write $\,(s,\TT c,\TT d) \rightarrow (s',\TT c',\TT d') \times \sigma\,$ the transitions, for readability).
\end{itemize}
A pointer machine will be called \emph{deterministic} if for any $\,A \,\in\, \TT S\times \Sigma\times \IO\,$, there is at most one $\,B\,\in\, \TT S\times \Sigma\times \IO\,$ and one $\,\sigma\in \F S_N\,$ such that $\,A\rightarrow B \times \sigma\,\in\,\Delta\,$.
In that case we can see $\,\Delta\,$ as a partial function, and we say that $\,M\,$ is \emph{reversible} if $\,\Delta\,$ is a partial injection.
}
We call the first of the $\,N\,$ pointers the \emph{main} pointer, it is the only one that can move.
The other pointers are referred to as the \emph{auxiliary} pointers.
An auxiliary pointer will be able to become the main pointer along the computation thanks to permutations.

\define[configuration]
{Given the length $\,n\,$ of a word $\,W=\start\TT c_1\dots \TT c_n\,$ over $\,\Sigma\,$ and a pointer machine $\,M=(N,\TT S,\Delta)\,$, a \emph{configuration} of $\,(M,n)\,$ is an element of 
\[\,\TT S\times\Sigma\times\IO\times\{0,1,\dots,n\}^N\,\]}
\vspace{-5mm}
%Compacte
The element of $\,\TT S\,$ is the state of the machine and the element of $\,\Sigma\,$ is the letter the main pointer points at.
The element of $\,\IO\,$ is the direction of the next move of the main pointer, and the elements of $\,\{0,1,\dots,n\}^N\,$ correspond to the positions of the (main and auxiliary)  pointers on the input.

\smallskip

As the input tape is considered cyclic with a special symbol marking the beginning of the word (recall Definition \ref{words}), the pointer positions are integers \emph{modulo} $\,n+1\,$ for an input word of length $\,n\,$.

\define[transition]
{Let $\,W\,$ be a word and $\,M=(N,\TT S,\Delta)\,$ be a pointer machine.
A \emph{transition} of $\,M\,$ on input $\,W\,$ is a triple of configurations
\[s,\TT c,\TT d,(p_1,\dots,p_N) \trans{\TT{MOVE}}{} s,\TT c',\overline{\TT d},(p_1',\dots,p_N') \trans{\TT{SWAP}}{} s',\TT c'',\TT d',(p_{\sigma(1)}',\dots,p_{\sigma(N)}') \]
such that
\begin{enumerate}
	\item if $\,\TT d\in\IO\,$, $\,\overline{\TT d}\,$ is the other element of $\,\IO\,$,
	\item $p_1'=p_1+1\,$ if $\,\TT d=\i\,$ and $\,p_1'=p_1-1\,$ if $\,\TT d=\o\,$,
	\item $p_i'=p_i\,$ for $\,i\neq 1\,$,
	\item $\TT c\,$ is the letter at position $\,p_1\,$ and $\,\TT c'\,$ is the letter at position $\,p_1'\,$, \label{condition}
	\item and $(s,\TT c',\overline{\TT d}) \rightarrow (s',\TT c'',\TT d') \times \sigma\,$ belongs to $\,\Delta\,$.
\end{enumerate}
}
There is no constraint on $\,c''\,$, but every time this value differs from the letter pointed by $\,p_{\sigma(1)}'\,$, the computation will halt on the next \texttt{MOVE} phase, because there is a mismatch between the value that is supposed to have been read and the actual bit of $\,W\,$ stored at this position, and that would contradict the first part of item \ref{condition}.
In terms of wirings, the \texttt{MOVE} phase corresponds to the application of the representation of the word, whereas the \texttt{SWAP} phase corresponds to the application of the observation.

%Note that the maintenance of $\,\TT c''\,$ as the letter pointed to by the (new) main pointer is not required at the \texttt{SWAP} phase, which will cause most computation to stop unexpectedly.
%The point is that it is possible to implement that maintainance using the states of the machine (typically, add a store for the values of the pointers, which has finitely many configurations, to the state of the machine), and therefore we favor the simpler and more liberal definition.

\define[acceptation]
{\label{translate}We say that $\,M\,$ accepts $\,W\,$ if any sequence of transitions $\,\big(C_i\trans{\TT{MOVE}}{}C_i'\trans{\TT{SWAP}}{}C_i''\big)\,$ such that $\,C''_i=C_{i+1}\,$ for all $\,i\,$ is necessarily finite.

We write $\,\lang(M)\,$ the set of words accepted by $\,M\,$.}

This means informally: we consider that a pointer machine accepts a word if it cannot ever loop, from whatever configuration it starts from.
%To retrieve computation as one could expect it, it is necessary to write programs with a lot of guesses on the values to be read, to store some information in the configurations and to keep in mind that the number of configurations can always be bounded.
That a lot of paths of computation accepts \enquote{wrongly} is no worry, since only rejection is meaningful: our pointer machines compute in a \enquote{universally non-deterministic} way to stick to the acceptance condition of wirings, nilpotency.

%Of course, this rather unusual acceptance condition is given with the acceptance condition of wirings in mind: nilpotency.
%As a side-effect, it gives evidences that our pointer machines compute in a \enquote{universally non-deterministic} way.

%This computational behaviour is suited to be embeded into the wirings, it reflects their wild way of computing, massively non-deterministic: 

\proposition[space and pointer machines]
{\label{pointerl}If $\,L\in \text{\sc NLogspace}\,$, then there exist a pointer machine $\,M\,$ such that $\,\lang(M)=L\,$.
Moreover, if $\,L\in \text{\sc Logspace}\,$ then $\,M\,$ can be chosen to be reversible.
}
\proof[sketch]{%
It is well-known that read-only Turing Machines --~or equivalently (non-)Deterministic Multi-Head Finite Automata~-- characterize {\sc (N)Logspace} \cite{Hartmanis1972}.
It takes little effort to see that our pointer machines are just a reasonable rearrangement of this model, since it is always possible to encode the missing information in the states of the machine.
%Multi-Head Finite Automata: that only one pointer may move does not lower the computational power of our device since it is always possible to switch the rôle of the pointers.
%We can get back to the expected notion of transition by encoding into the states the values that can be read by the pointers.

That acceptation and rejections are \enquote{reversed} is harmless in the deterministic (or equivalently reversible \cite{Lange2000}) case, and uses that {\sc co-NLogspace = NLogspace} to get the expected result in the non-deterministic case.
}

\bigskip
As we said, our pointer machines are designed to be easily simulated by wirings, so that we get the expected result almost for free.

\theorem[space completeness]
{If $\,L\in \text{\sc NLogspace}\,$, then there exist an observation $\,\phi\in\C O_\Sigma\,$ such that $\,\lang(\phi)=L\,$.
Moreover, if $\,L\in \text{\sc Logspace}\,$ then $\,\phi\,$ is an isometric wiring.
}

\proof{%
By Proposition \ref{pointerl}, there exists a pointer machine $\,M=(N,\TT S,\Delta)\,$ such that $\lang(M) = L$.
We associate to $\,S\,$ a set of distinct closed terms $\,[\TT S]\,$ and write $\,[s]\,$ the term associated to $\,s\,$.
To any element $\,D=(s,\TT c,\TT d) \rightarrow (s',\TT c',\TT d') \times \sigma\,$ of $\,\Delta\,$ we associate the flow 
\[[D]:=([s']\p\TT c'\p\TT d' \flow [s]\p\TT c\p\TT d) \tensor \sigma \:\in[\TT S]\tensor\Sigma\tensor\IO\tensor\C S_n\,\]
and we define the wiring $\,[M]\in\C O_\Sigma^+\,$ as $\,\displaystyle\sum_{D\in \Delta} [D]\,$.

One can easily check that this translation preserves the language recognized and relates reversibility with isometricity: in fact, $\,M\,$ is reversible if and only if $\,[M]\,$ is an isometric wiring.
Then, if $\,L\in \text{\sc Logspace}\,$, $\,M\,$ is deterministic and can always be chosen to be reversible \cite{Lange2000}.
}
