We are going to define how observations accept and reject words.
This needs a discussion, because there is an issue with word representations: an observation is an element of $\,\ualg\,$ and can therefore only interact with \emph{representations} of a word, and there are many possible representation of the same word (in Definition \ref{words}, different choices of closed terms lead to different representations). Therefore one has to ensure that acceptance or rejection is independent of the representation, so that the notion makes the intended sense.

The termination of computations will correspond to the algebraic notion of \emph{nilpotency}, which we recall here.

\define[nilpotency]
{A wiring $\,F\,$ is \emph{nilpotent} if $\,F^n=0\,$ for some $\,n\,$.}
%\vspace{-2mm}

\define[automorphism]
	{An \emph{automorphism} of a $\ast$-algebra $\,\C A\,$ is a linear application $\,\varphi\,:\: \C A\rightarrow\C A\,$ such that for all~$\,F,G\in\C A\,$: $\:\varphi(FG)=\varphi(F)\varphi(G)\,$, $\,\varphi(F^\dagger)=\varphi(F)^\dagger\,$ and $\,\varphi\,\text{ is injective}\,$.

%	If there is a partial isometry $\,U\in \C A\,$ such that $\,\varphi(a)=U^\dagger aU\,$ for all $\,a\,$, then we say that $\,\varphi\,$ is an \emph{inner} automorphism.\nolinebreak
}

\example{$\,\varphi(U_1\ptensor U_2):=U_2\ptensor U_1\,$ defines an automorphism of $\,\ualg \ptensor\ualg\,$.}

\smallskip
\notation{If $\,\varphi\,$ is an automorphism of $\,\C A\,$ and $\,\psi\,$ is an automorphism of $\,\C B\,$, we write $\,\varphi\ptensor\psi\,$ the automorphism of $\,\C A\ptensor\C B\,$ defined for all $\,A\in \C A, B\in \C B\,$ as $\,(\varphi\ptensor\psi)(A\ptensor B):=\varphi(A)\ptensor\psi(B)\,$.}

\define[normative pair]
	{\label{normpair}A pair $\,(\C A,\C B)\,$ of $\ast$-algebras is a \emph{normative pair} whenever any automorphism $\,\varphi\,$ of $\,\C A\,$ can be extended into an automorphism $\,\overline\varphi\,$ of the $\ast$-algebra $\,\C E\,$ generated by $\,\C A \cup\C B\,$ such that $\,\overline\varphi(F)=F\,$ for any $\,F\in \C B\,\subseteq\,\C E\,$.
}
%\vspace{-1mm}
The two following propositions set the basis for a notion of acceptance/rejection independent of the representation of a word.
%\vspace{-1mm}
%\smallskip
%\notation{Given a finite set of closed terms $\,S\,$, we write $\,\displaystyle1_S:=\sum_{\TT s\in S} s \flow s\,$.}

\proposition[automorphic representations]
	{\label{automorphic}Any two representations $\,W(t_i),W(u_i)\,$ of a word $\,W\,$ over $\,\Sigma\,$ are automorphic: there exists an automorphism $\,\varphi\,$ of $\,(\closed)^1\,$ such that \[\,(\Id_{\,\ualg}\ptensor\varphi)\big(W(t_i)\big)=W(u_i)\,\]
}
\vspace{-8mm}
\proof{Consider a bijection $\,f\,:\,\closed \rightarrow\closed\,$ such that $\,f(t_i)=u_i\,$ for all $\,i\,$. Then set $\,\varphi(v\p x\flow w\p x):=f(v)\p x\flow f(w)\p x\,$, extended by linearity. 
}
\vspace{-1mm}
\smallskip
\proposition[nilpotency and normative pairs]
	{\label{normative}Let $\,(\C A,\C B)\,$ be a normative pair, $\,\varphi\,$ an automorphism of $\,\C A\,$,
	$\,F\in \ualg\ptensor\C A\,$, $\,G\in \ualg\ptensor\C B\,$ and let $\,\psi:=\Id_\ualg\ptensor \varphi\,$. Then $\,GF\,$ is nilpotent if and only if $\,G\,\psi(F)\,$ is nilpotent.%
}
\vspace{-2mm}
\proof{Let $\,\overline\varphi\,$ be the extension of $\,\varphi\,$ as in Definition \ref{normpair} and $\,\overline\psi:=\Id_{\,\ualg}\ptensor\overline\varphi\,$.

We have for all $\,n\neq 0\,$ that $\,(G\psi(F))^n=(\overline\psi(G)\overline\psi(F))^n=(\overline\psi(GF))^n=\overline\psi((GF)^n)\,$.

By injectivity of $\,\overline\psi\,$, $\,(G\psi(F))^n=0\,$ if and only if $\,(GF)^n=0\,$.
}

\corollary[independance]
{\label{indep}Let $\,\big((\C \closed)\tpower 1,\C B\big)\,$ be a normative pair and $\,W\,$ a word over $\,\Sigma\,$.

Let $\,F\in (\closed^\ast\ptensor\Sigma^\ast\ptensor\IO^\ast)\ptensor\C B\,$ be an wiring, then the product $\,FW(t_i)\,$, is nilpotent for one choice of $\,(t_i)\,$ if and only if it is nilpotent for all choices of~$\,(t_i)\,$.}

%Now we can use the tools from section \ref{actions} to build a particular normative pair.
%\pagebreak[3]

The word and observation algebras we introduced earlier can be shown to form a normative pair.

\theorem%[a normative pair]
{The pair $\,\big((\closed)\tpower1,\C S\big)\,$ is normative.}

\proof[sketch]{By simple computations, the set
\[\,\C A:=\vect\set{\sigma F}{\sigma \in \C S \mbox{ and }\, F\in (\closed)^\infty}\,\]
 can be shown to be a $\ast$-algebra $\,\C E\,$, the $\ast$-algebra generated by $\,\C S\cup(\closed)\tpower 1\,$.

Now, if $\,\varphi\,$ is an automorphism of $\,(\closed)\tpower 1\,$, it can be written as $\,\varphi(G\ptensor 1)=\psi(G)\ptensor 1\,$ for all $\,G\,$, with $\,\psi\,$ an automorphism of $\,\closed^\ast\,$.

We set for $\,F=F_1\ptensor\cdots\ptensor F_n\ptensor 1\in(\closed)\tpower n\,$, $\,\tilde\varphi(F):=\psi(F_1)\ptensor\cdots\ptensor \psi(F_n)\ptensor 1\,$ which extends into an automorphism of $\,(\closed)\tpower \infty\,$ by linearity.
Finally, we extend $\,\tilde\varphi\,$ to $\,\C A\,$ by $\,\overline\varphi(\sigma F):=\sigma\,\tilde\varphi(F)\,$.  It is then easy to check that $\,\overline\varphi\,$ has the required properties.
}

\medskip
\remark{Here we sketched a direct proof for brevity, but this can also be shown by involving a little more mathematical structure (actions of permutations on the unbounded tensor and crossed products) which would give a more synthetic proof.}

\medskip
We can then define the notion of the language recognized by an observation, \textit{via} Corollary \ref{indep}.

\define[language of an observation]
	{Let $\,\phi\in\C O_\Sigma^+\,$ be an observation over $\,\Sigma\,$.
	The \emph{language recognized by $\,\phi\,$} is the following set of words over $\,\Sigma\,$
	\vspace{-1mm}
	\[\lang(\phi):=\set{W\:\mbox{word over}\,\Sigma}{\phi\,W(t_i)\,\mbox{nilpotent for any choice of }\,(t_i)}\]
	}
	\vspace{-10mm}
	% Compacte
