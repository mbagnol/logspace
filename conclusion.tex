%A pending question about this approach to complexity classes is about the structure that are exactly needed to carry on the construction. 
A pending question about this approach to complexity classes is to delimit the minimal prerequisites of the construction, its core.

Earlier works \cite{girard_normativity_2012,aubert_characterizing_2012,seiller_logarithmic_2013} made use of von Neumann algebras to get a setting that is expressive enough, we ligthen the construction by using simpler objects.
Yet, the possibility of representing the action of permutations on a unbounded tensor product is a common denominator that seems deeply related to logarithmic space and pointer machines.
%Earlier works made use of von Neumann algebras to get a setting that is expressive enough, which is not the case in this article.
%On the contrary, the possibility of representing the action of permutations on a unbounded tensor product is a common denominator of
%\cite{girard_normativity_2012,aubert_characterizing_2012,seiller_logarithmic_2013} and our work.
%It seems more deeply related to logarithmic space and pointer machines.

\smallskip
The language of the unification algebra gives us a twofold point of view on computation, either through algebraic structures (that are described finitely by wirings) or pointer machines.
We may therefore start exploring possible variations of the construction, combining intuitions from both worlds.

\smallskip
For instance, the choice of a normative pair can affect the expressivity of the construction:
the more restrictive the notion of representation of words is, the more liberal that of observation can become.
Whether and how this can affect the corresponding complexity class is definitely a direction for future work.

\smallskip
The logical counterpart of this work also needs clarifying.
Indeed, the idea of representation of words comes directly from proof-theory, while the notion of observation does not correspond to any known logical construction.

\smallskip
Finally, as the execution in our setting is the product of matrices, it seems possible to relate our modelisation with parallel computation. Indeed, the product of matrices is known to be computable efficiently on a parallel machine, and so is matching of linear terms.
%Yet, we did not take full profit of this idea, altough it seems more easy to express it with unification rather than with von Neumann algebras.
