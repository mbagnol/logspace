

\smallskip

An important point that needs to be clarified is 



%With respect to the earlier \cite{girard_normativity_2012,aubert_characterizing_2012,seiller_logarithmic_2013},
%this work begins to clarify one point: what is really needed to carry on the construction that captures logarithmic space computation?

%Indeed, these earlier works used in place of the unification algebra the so-called \emph{hyperfinite factor},
%which involved advanced notions of von~Neumann algebras theory.
%Our work shows that the von Neumann structure is indeed not indispensible,
%whereas the ability to represent the action of permutation groups on an unbounded tensor product seems to be an important element in the relation to pointer machines.






\smallskip
The language the unification algebra gives us a twofold point of view on computation, either through algebraic structures or pointer machines. We can therefore start exploring possible variations of the construction, combining intuitions from both points of view.

%For instance, a natural operation one would like to define on pointer machines would be the one the resets the main pointer to the initial position holding the symbol $\,\star\,$.
%This is not possible within the setting of this article, because of the notion of normative pair:
%this would require to know in advance the term $\,t_0\,$ corresponding to the initial position.

\smallskip
For instance, the choice of a normative pair can affect the expressivity of the construction:
the more restrictive the notion of representation of word is, the more liberal that of observation can become.
Whether and how this can affect the corresponding complexity class is definitely a direction for future work.

\smallskip
The logical counterpart of this work also need clarifying. Indeed, the notion of representation of word comes directly from proof-theory, while the notion of observation does not correspond to any known logical construction.
