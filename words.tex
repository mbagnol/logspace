The representation of words over an alphabet in the unification algebra directly comes from the translation of Church lists in linear logic and their interpretation in Geometry of Interaction models~\cite{girard_geometry_1989,girard_geometry_1995}.

This proof-theoretic origin is an useful guide for intuition, even if we give here a more straightforward definition of the notion.

\medskip
From now on, we fix a set of two distinguished constant symbols $\,\IO:=\{\:\o,\i\:\}\,$.

\define[word algebra]
	{To a set $\,S\,$ of closed terms, we associate the $\ast$-algebra
	
	\vbox{
	\[S^\ast:= \vect\set{t\flow u}{t,u\in S}\]
	
	\hfill{\it\scriptsize(which is indeed an algebra because unification of closed terms is simply equality)}
	}
	
	\smallskip
	%When it is clear from the context, we will simply write $\,\Sigma\,$ in place of $\,[\Sigma]\,$.
	The \emph{word algebra} associated to a finite set of constant symbols $\,\Sigma\,$ is the $\ast$-algebra defined as 
	
	\vbox{
	\[\,\C W_\Sigma:=(\unitalg\tensor\Sigma^\ast\tensor\IO^\ast)\tensor(\closed^\ast)\tpower 1\,\]

	\vspace{-1mm}
	\hfill{\it\scriptsize(remember that $\,\closed\,$ is the set of \emph{all} closed terms. $(.)\tpower 1$ is the case $n=1$ of Definition \ref{unbounded})}
	}
	
	\vspace{-2mm}
}

The words we consider are cyclic, with a begin/end marker $\,\start\,$, a reserved constant symbol.
For example the word $\,\TT{0010}\,$ is to be thought of as $\,\start\TT{0010}=\TT{10}\!\start\!\TT{00}=\TT0\!\start\!\TT{001}=\cdots\:$.

We consider therefore that the alphabet $\,\Sigma\,$ always contains the symbol $\,\start\,$.

\define[word representation]
	{\label{words}Let $\,W=\start\TT c_1\dots \TT c_n\,$ be a word over $\,\Sigma\,$ and $\,t_0,t_1,\dots,t_n\,$ be distinct closed terms.
	
	The \emph{representation} $\,W(t_0,t_1,\dots,t_n)\in \C W_\Sigma^\concrete\,$
	with respect to $\,t_0,t_1,\dots,t_n\,$
	of $\,W\,$ is an isometric wiring (Definition~\ref{concrete}), defined as
	\vspace{-1mm}	
\[
	\begin{array}{ccl}
		W(t_0,t_1,\dots,t_n) :=		& &x\p\start\p\i\p (t_0\p y) \sflow x\p\TT c_1\p\o\p (t_1\p y) \\
				&+& x\p\TT c_1\p\i\p (t_1\p y) \sflow x\p\TT c_2\p\o\p (t_2\p y) \\
				&+&\ \cdots\ \\
				&+& x\p\TT c_n\p\i\p (t_n\p y) \sflow x\p\start\p\o\p (t_0\p y)
	\end{array}
\]
\vspace{-5mm}
}
